\PassOptionsToPackage{svgnames}{xcolor}
\documentclass[a5paper]{extarticle}
\usepackage[british]{babel}
%\usepackage[papersize={150mm,123mm},hmargin=10mm,vmargin={10mm,15mm}]{geometry}
\usepackage[hmargin=10mm,vmargin={10mm,15mm}]{geometry}

\usepackage{noto}
\usepackage{fontawesome5}
\usepackage{microtype}
\usepackage{setspace}
\setstretch{1.2}

\usepackage{parskip}
\newlength{\currentparskip}
\setlength{\currentparskip}{\parskip}

\setcounter{tocdepth}{1}

\usepackage{xltabular}
\renewcommand{\arraystretch}{1.5}
\usepackage{adjustbox}
\usepackage{csquotes}
\usepackage{multicol}
\usepackage[mla]{xellipsis}
\usepackage{tgm1}
\tikzset{tgmnode/.append style={draw=DarkRed}}
\usetikzlibrary{graphdrawing, calc, arrows.meta}
\usegdlibrary{layered}
\tikzset{
  forked/.style={%
    to path={|- ($(\tikztotarget) + (0,#1)$) -- (\tikztotarget) \tikztonodes}},
  forked/.default={0.5},
  hooked/.style={%
    rounded corners,
    to path={-- ($(\tikztotarget) + (0,#1)$) -- (\tikztotarget) \tikztonodes}},
  hooked/.default={1}
}
\usepackage[colorlinks]{hyperref}
\usepackage[nameinlink]{cleveref}

\title{Cragne Manor: A Multilayered Walkthrough for the Spoiler Averse}
\author{Alex Ball}
\date{2019}

\begin{document}
\thispagestyle{empty}
\vspace*{\stretch{2}}
\begin{center}
\Huge\textsc{Cragne Manor}

\huge A Multilayered Walkthrough for the Spoiler Averse\par

\bigskip
\Large Alex Ball\par
\end{center}
\vspace*{\stretch{3}}
{\centering\Large Version 1.0 --- \today\par}
\newpage

\tableofcontents

\newpage
\section{About this document}\label{sec:about}

This is not a regular walkthrough.
It does not provide a turn-by-turn set of commands for you to feed into the game.
There are two reasons for this.
First, that would take away the fun of working things out for yourself.
Second, there's so much in this game that I'm sure I've missed some gems,
and I don't want to make you miss them too.

I have therefore arranged this walkthrough in sections so you can choose how much help you expose yourself to.
My recommended route in \cref{sec:route} should be relatively spoiler-free.
It just names a room to tackle and then the direction to go in to reach the next location.
Some sets of directions have been marked with an asterisk;
the significance of this may or may not become obvious to you as you play.
Where the action is concentrated in a particular area,
a map of the ‘open’ rooms in that area is provided.

My route is designed to minimize the travelling you have to do,
rather than follow what you might call story branches.
If you would rather work out a route for yourself,
I have provided a set of maps for you in \cref{sec:full-map}.
The first page shows the geographical layout of the rooms,
while the second shows you the order in which the rooms must be completed.
The latter makes it a bit more obvious what you are looking for in each room,
or at least where the thing you have just found or discovered should be used.
This may save you some unnecessary traipsing around,
but equally it may take away some of the fun.

If you want some help, the steps in the recommended route link through to a
corresponding room description in \cref{sec:sum};
this will tell you a bit more about what you are supposed to be doing,
but won't tell you how to do it.
If you get stuck, you can follow the link through to the corresponding room solution in \cref{sec:sol};
this will describe the steps you have to go through,
but even here I have tried to be a little vague so I don't spoil all the surprises for you.

Finally, \cref{sec:nb} contains various notes and clues that didn't belong
in the other sections.

I don't recommend you print this document out.
I try to keep information on different subjects on different pages,
so you don't accidentally see something you don't want to.
There is a \emph{lot} of wasted white space.

\newpage
\subsection{Terminology}\label{sec:terms}

Rooms in this game may be solved by doing one or more of the following:

\begin{itemize}
\item
  unblocking an exit;
\item
  obtaining an item you'll need in another room;
\item
  learning a critical piece of information.
\end{itemize}

You cannot solve all the rooms in the game the first time you encounter them.
This is reflected in this walkthrough in the following way:

\begin{itemize}
\item
  If you just need to pass through, it is omitted from the route and marked
  on the map in red.
\item
  If it is possible for you to work out what sort of item or clue you need to
  solve a puzzle in a room (other than a key), it is included in the route
  and marked with ‘establish requirements’. It is marked on the map in red.
\item
  If it is possible to solve some but not all of the challenges in a room,
  it is marked in the route with ‘partial’ and marked on the map in amber.
\item
  If it is possible to solve all of the challenges in a room, it is included
  in the route unmarked and marked on the map in green.
\item
  If there are some interesting things to examine or do in the room, but not
  one of the above activities, it is marked in the route as ‘optional’ and
  marked on the map in green.
\end{itemize}

\bigskip
\begin{adjustbox}{center}
\sffamily
\begin{gamemap}[set grid={10em}{6em}]
\graph [no placement] { [y=0]
  "Room you\\can leave\\unsolved" [x=0],
  "Room you\\can solve\\partially" [x=1,draw=DarkOrange],
  "Room you\\can solve\\fully" [x=2,draw=Green],
};
\end{gamemap}
\end{adjustbox}

\bigskip
\subsection{Corrections and additions}

If you find this walkthrough is incorrect, or if you want to contribute
missing information, please let me know via the GitHub issue tracker:
\url{https://github.com/alex-ball/cragne-manor-walkthrough/issues}.

\newpage
\section{General tips when playing the game}

I was interested in playing \emph{Cragne Manor} because I wanted a game I could dip in and out of,
and a game consisting of mainly one-room puzzles sounded ideal.
Even though the rooms weren't quite as independent as I'd thought,
they were independent enough that I didn't feel I had to scrutinize and memorize every detail.

There are some innovative features of the game that make it particularly playable.
One is the rather nice (and generous) form of inventory management.
Another is that, along the way, you will learn how to tell if you have everything you need to solve a particular room.
A particularly humane feature is the \textsc{take back} instruction,
which can be used to get you out of an unwinnable state where \textsc{undo} wouldn't.
This means you can try risky solutions with some degree of security,
and enjoy the many ways of dying or otherwise losing the game
without losing the ability to complete it.

Nevertheless, I recommend saving before beginning (or after finishing) each room,
as some non-fatal but irksome things may happen to you that you may not be able to undo easily.
Lastly, some other words of advice: examine everything, check your inventory regularly
(it can change without you being aware of it), don't be afraid to act on even surreal hints,
but also hold your nerve: some correct paths look highly unpromising.

Good luck!

\newpage
\section{Recommended route}\label{sec:route}

\begin{description}
\item[{\hyperref[sec:req-Railway-Platform-0]{Railway Platform}}] (establish requirements)\\
  \emph{Go S}
\item[{\hyperref[sec:req-Train-Station-Lobby-0]{Train Station Lobby}}] (partial)\\
  \emph{Go W}
\item[{\hyperref[sec:req-Train-Station-Restroom]{Train Station Restroom}}] ~\\
  \emph{Go E, S}
\item[{\hyperref[sec:req-Exterior-of-Train-Station]{Exterior of Train Station}}] ~\\
  \emph{Go S}
\item[{\hyperref[sec:req-Milkweed]{Milkweed}}] ~\\
  \emph{Go S, NE}
\item[{\hyperref[sec:req-Churchyard]{The Churchyard}}] ~\\
  \emph{Go in}
\item[{\hyperref[sec:req-Mausoleum]{Mausoleum}}] ~\\
  \emph{Go out, NE}
\end{description}

\vspace*{\stretch{1}}
\begin{adjustbox}{center}
\sffamily\footnotesize
\begin{gamemap}[set grid={10em}{5em}]
\graph [no placement] { [x=0]
  "Railway\\platform" [entry] -- [going=s]
  lobby/"Train station\\lobby" [y=-1,exit=e,draw=DarkOrange] -- [going=w]
  "Train station\\restroom" [x=-1,y=-1,draw=Green],
  lobby -- [going=s]
  "Exterior of\\Train station" [y=-2,draw=Green]  -- [going=s]
  "Milkweed" [y=-3,draw=Green] -- [going=s]
  "Church\\Exterior" [y=-4,exit=e,tunnel] -- [going=ne]
  cy/"The\\Churchyard" [x=1,y=-3,exit=ne,draw=Green] -- [going=in]
  "Mausoleum" [x=2.1,y=-3,draw=Green]
};
\end{gamemap}
\end{adjustbox}

\vspace*{\stretch{1}}
\newpage\phantomsection\label{sec:route-1}
\begin{description}
\item[{\hyperref[sec:req-The-Dim-Recesses-of-the-Forest]{The Dim Recesses of the Forest}}] ~\\
  \emph{Go N, S, SE}
\item[{\hyperref[sec:req-Shack-Exterior-0]{Shack Exterior}}] (partial)\\
  \emph{Go SE}
\item[{\hyperref[sec:req-Outside-the-Library]{Outside the Library}}] (optional)\\
  \emph{Go E}
\item[{\hyperref[sec:req-Public-Library-0]{Backwater Public Library}}] (establish requirements)\\
  \emph{Go W, W}
\item[{\hyperref[sec:req-Estate-Agent's-Office]{Estate Agent's Office}}] ~\\
  \emph{Go E, S}
\item[{\hyperref[sec:req-Town-Square-0]{Town Square}}] (establish requirements)\\
  \emph{Go SE}
\item[{\hyperref[sec:req-River-Walk]{River Walk}}] ~\\
  \emph{N}
\item[{\hyperref[sec:req-Under-the-Bridge-0]{Under the Bridge}}] (partial)\\
  \emph{Go S -- Go NW, W, N, N, N, N*}
\end{description}

\vspace*{\stretch{1}}
\begin{adjustbox}{max width=\textwidth,center}
\sffamily\footnotesize
\begin{gamemap}
\graph[no placement] {
  ch/"Church\\Exterior" [x=0,y=-4,exit=n,entry] -- [going=ne]
  cy/"The\\Churchyard" [x=1,y=-3,draw=Green] -- [going=in]
  "Mausoleum" [x=2.1,y=-3,draw=Green],
  cy  -- [going=ne]
  dr/"The Dim\\Recesses of\\the Forest" [x=2,y=-2,draw=Green] -- [going=se]
  "Shack\\Exterior" [x=3,y=-3,draw=DarkOrange] -- [going=se]
  ol/"Outside\\the Library" [x=4,y=-4,tunnel,draw=Green] -- [going=e]
  "Backwater\\Public\\Library" [x=5.1,y=-4],
  ol -- [going=w]
  "Estate\\Agent's\\Office" [x=3,y=-4,draw=Green],
  ol -- [going=s]
  ts/"Town\\Square" [x=4,y=-5,exits={e,sw}] -- [going=se]
  "River\\Walk" [x=5,y=-6,draw=Green,tunnel] -- [going=n]
  ub/"Under the\\Bridge" [x=5,y=-5,exit={d=\tgmSE},draw=DarkOrange],
  dr -- [going=n]
  "The Old Well" [x=2,y=-1,tunnel],
  ts -- [going=w] ch
};
\end{gamemap}
\end{adjustbox}

\vspace*{\stretch{1}}
\newpage\phantomsection\label{sec:route-2}
\begin{description}
\item[{\hyperref[sec:req-Railway-Platform]{Railway Platform}}] ~\\
  \emph{Go S, S, S, S, NE, NE, N}*
\item[{\hyperref[sec:req-The-Old-Well]{The Old Well}}] ~\\
  \emph{Go down}
\item[{\hyperref[sec:req-Circular-Room]{Circular Room}}] ~\\
  \emph{Go up -- Go S, SE, SE* -- Go S, E}
\item[{\hyperref[sec:req-Bridge]{Bridge}}] ~\\
  \emph{Go E}
\item[{\hyperref[sec:req-Outside-Pub]{Outside Pub}}] ~\\
  \emph{Go E}
\item[{\hyperref[sec:req-Constabulary-Road]{Constabulary Road}}] (optional)\\
  \emph{Go N}
\item [{\hyperref[sec:req-Backwater-Jail]{Backwater Jail} --
        \hyperref[sec:req-Padded-Cell]{Padded Cell} --
        \hyperref[sec:req-Backwater-Jail]{Backwater Jail}}] ~\\
  \emph{Go S, NE, SE}
\end{description}

\vspace*{\stretch{1}}
\begin{adjustbox}{max width=\textwidth,center}
\sffamily\footnotesize
\begin{gamemap}[set grid={8em}{6em}]
\graph [no placement] {
  ub/"Under the\\Bridge" [x=5,y=-5.25,exit={d=\tgmSE},draw=DarkOrange] -- [going=s]
  "River\\Walk" [x=5,y=-6,entry,draw=Green] -- [going=nw]
  ts/"Town\\Square" [x=4,y=-5,exits=sw] -- [going=w]
  ch/"Church\\Exterior" [x=2.5,y=-5,tunnel] -- [going=n,dashed]
  "Railway\\Platform" [x=2.5,y=-3,tunnel,draw=Green],
  ch -- [go={ne}{s},dashed]
  ow/"The Old\\Well" [x=3.5,y=-3,tunnel,draw=Green] -- [go={d=\tgmENE}{u=\tgmWSW}]
  "Circular\\Room" [x=4.5,y=-3,draw=Green],
  ow -- [go={s}{nw},dashed]
  ol/"Outside\\the Library" [x=4,y=-4,exit=w,tunnel,draw=Green] -- [going=s]
  ts -- [going=e]
  "Bridge" [x=5,y=-4.7,draw=Green] -- [going=e]
  "Outside\\Pub" [x=5.9,y=-5,exit=n,draw=Green] -- [going=e]
  "Constabulary\\Road" [x=7,y=-5,exit=ne,tunnel,draw=Green] -- [going=n]
  "Backwater\\Jail" [x=7,y=-4,draw=Green] -- [going=n]
  "Padded\\Cell" [x=7,y=-3,draw=Green],
  ol  -- [going=e]
  "Backwater\\Public\\Library" [x=5.1,y=-4]
};
\end{gamemap}
\end{adjustbox}

\vspace*{\stretch{1}}
\newpage\phantomsection\label{sec:route-3}
\begin{description}
\item[{\hyperref[sec:req-Outside-the-Plant]{Outside the Plant}}] ~\\
  \emph{Go in, up}
\item[{\hyperref[sec:req-Dusty-Office]{Dusty Office}}] ~\\
  \emph{Go down, W}
\item[{\hyperref[sec:req-Bathroom-of-the-Meatpacking-Plant]{Bathroom of the Meatpacking Plant}}] ~\\
  \emph{Go E, out, NW}
\item[{\hyperref[sec:req-Hillside-Path]{Hillside Path}}] (partial)\\
  \emph{Go N}
\item[{\hyperref[sec:req-Front-Walk]{Front Walk}}] (optional)\\
  \emph{Go NE}
\item[{\hyperref[sec:req-Cragne-Family-Plot]{Cragne Family Plot}}] (partial)\\
  \emph{Go NW}
\item[{\hyperref[sec:req-Back-Garden]{Back Garden}}] ~\\
  \emph{Go in}
\item[{\hyperref[sec:req-Shambolic-Shack-0]{Shambolic Shack}}] (partial)\\
  \emph{Go out, SW, in, out, SE -- Go S, SW, W, W, W, SE* -- Go N}
\end{description}

\vspace*{\stretch{1}}
\begin{adjustbox}{max width=\textwidth,center}
\sffamily\footnotesize
\begin{gamemap}
\graph[no placement] {
  "Constabulary\\Road" [x=7,y=-5,exits={n,w},entry,draw=Green] -- [going=ne]
  hp/"Hillside\\Path" [x=8,y=-4,draw=DarkOrange] -- [going=se]
  "Outside\\the Plant" [x=9,y=-5,tunnel,draw=Green] -- [going=in]
  mp/"The\\Meatpacking\\Plant" [x=10.5,y=-6] -- [going=u]
  "Dusty\\Office" [x=10.5,y=-5,draw=Green],
  mp -- [going=w]
  "Bathroom of\\the Meatpacking\\Plant" [x=9,y=-6,draw=Green],
  hp -- [going=n]
  fw/"Front\\Walk" [x=8,y=-3,exit=n,tunnel,draw=Green] -- [going=nw]
  og/"Outside the\\Greenhouse" [x=7,y=-2,draw=Green] -- [going=ne]
  bg/"The Cragne\\Manor's Back\\Garden" [x=8,y=-1,draw=Green] -- [going=se]
  "Cragne\\Family\\Plot" [x=9,y=-2,exit=in,draw=DarkOrange] -- [going=sw]
  fw,
  og -- [go={in=\tgmWSW}{o=\tgmENE}]
  "Greenhouse"[x=5.9,y=-2],
  bg -- [going=in]
  "The\\Shambolic\\Shack" [x=9.5,y=-1,draw=DarkOrange]
};
\end{gamemap}
\end{adjustbox}

\vspace*{\stretch{1}}
\newpage\phantomsection\label{sec:route-4}
\begin{description}
\item[{\hyperref[sec:req-Under-the-Bridge]{Under the Bridge}}] ~\\
  \emph{Go down, NW}
\item[{\hyperref[sec:req-Tunnel-Entrance]{Tunnel Entrance}}] ~\\
  \emph{Go NE}
\item[{\hyperref[sec:req-Small-Chamber]{Small Chamber}}] ~\\
  \emph{Go SW, U, U}
\item[{\hyperref[sec:req-Tiny-Windowless-Office]{Tiny Windowless Office}}] (partial)\\
  \emph{Go down, down, SE, up, S -- Go NW, E, E, E, NE, N* -- Go N}
\end{description}

\vspace*{\stretch{1}}
\begin{adjustbox}{center}
\sffamily\footnotesize
\begin{gamemap}
\graph [no placement] {
  "River\\Walk" [x=0,y=-8,draw=Green,entry,exit=nw] -- [going=n]
  "Under the\\Bridge" [x=0,y=-7,draw=Green] -- [go={d=220}{u}]
  "Subterranean\\tunnel" [x=-1,y=-8,exit=e] -- [going=nw]
  te/"Tunnel\\Entrance" [x=-2,y=-7,draw=Green] -- [going=ne]
  "Small\\Chamber" [x=-1,y=-6,draw=Green],
  te -- [going=u]
  "Basement" [x=-2,y=-6] -- [going=u]
  "Tiny Window-\\less Office" [x=-2,y=-5,draw=DarkOrange],
};
\end{gamemap}
\end{adjustbox}

\vspace*{\stretch{1}}
\newpage\phantomsection\label{sec:route-5}
\begin{description}
\item[{\hyperref[sec:req-Mudroom]{Mudroom}}] ~\\
  \emph{Go N}
\item[{\hyperref[sec:req-Foyer]{Foyer}}] (optional)\\
  \emph{Go N}
\item[{\hyperref[sec:req-Landing-at-the-Bottom-of-Stairs]{Landing at the Bottom of Stairs}}] ~\\
  \emph{Go up}
\item[{\hyperref[sec:req-Top-of-Stairs]{Top of Stairs}}] ~\\
  \emph{Go E}
\item[{\hyperref[sec:req-Upstairs-Hall-N]{Upstairs Hall, North End}}] (establish requirements)\\
  \emph{Go N}
\item[{\hyperref[sec:req-Nursery--Hillside-Path--Carol's-Room]{Nursery -- Hillside Path -- Carol’s Room}}] ~\\
  \emph{Go S}
\item[{\hyperref[sec:req-Hallway-South]{Hallway South}}] (partial)\\
  \emph{Go E}
\item[{\hyperref[sec:req-Library]{Library}}] ~\\
  \emph{Go W, W}
\item[{\hyperref[sec:req-Study]{Study}}] ~\\
  \emph{Go E, N, W, D, S, E}
\end{description}

\vspace*{\stretch{1}}
\begin{adjustbox}{max width=\textwidth,center}
\sffamily\footnotesize
\begin{gamemap}
\graph [no placement] {
  { [x=1]
  hp/"Hillside\\Path" [y=2,tunnel,draw=DarkOrange] -- [going=n]
  fw/"Front\\Walk" [y=3,entry,exits={nw,ne},draw=Green] -- [going=n]
  "Mudroom" [y=4,draw=Green] -- [going=n]
  f/"Foyer" [y=5,draw=Green,exits={e,w}] -- [going=n]
  "Landing at\\the Bottom\\of Stairs" [y=6,exits={n},draw=Green]
  } -- [go={u=\tgmESE}{d=\tgmWNW}]
  "Top of\\Stairs" [x=2,y=5,draw=Green] -- [going=e]
  hn/"Upstairs\\Hall N" [x=3,y=5,exit=e] -- [going=s]
  hs/"Hallway\\South" [x=3,y=4,exit=s,draw=DarkOrange] -- [going=e]
  "Library" [x=4,y=4,draw=Green],
  hs-- [going=w]
  "Study" [x=2,y=4,draw=Green],
  hn -- [going=n]
  "Nursery" [x=3,y=6,tunnel,draw=Green]
};
\end{gamemap}
\end{adjustbox}

\vspace*{\stretch{1}}
\newpage\phantomsection\label{sec:route-6}
\begin{description}
\item[{\hyperref[sec:req-Gallery]{Gallery}}] ~\\
  \emph{Go N, E}
\item[{\hyperref[sec:req-Music-Room]{Music Room}}] ~\\
  \emph{Go W, S, W, W}
\item[{\hyperref[sec:req-Court-0]{Court}}] (partial) ~\\
  \emph{Go E, S, S, NE}
\item[{\hyperref[sec:req-Cragne-Family-Plot-1]{Cragne Family Plot}}] ~\\
  \emph{Go in}
\item[{\hyperref[sec:req-Crypt]{Crypt}}] ~\\
  \emph{Go out, SW -- Go S, SW, W, W, W* }
\end{description}

\vspace*{\stretch{1}}
\begin{adjustbox}{center}
\sffamily\footnotesize
\begin{gamemap}
\graph [no placement] {
  fw/"Front\\Walk" [x=0,y=0,tunnel,exits={nw,s},draw=Green] -- [going=n]
  "Mudroom" [x=0,y=1,draw=Green] -- [going=n]
  f/"Foyer" [x=0,y=2,draw=Green] -- [going=w]
  "Court" [x=-1,y=2,draw=DarkOrange],
  f -- [going=e]
  "Gallery" [x=1,y=2,draw=Green] -- [going=n]
  "Rec\\Room" [x=1,y=3] -- [going=e]
  "Music\\Room" [x=2,y=3,draw=Green],
  f -- [going=n]
  "Landing at\\the Bottom\\of Stairs" [x=0,y=3,entry,exits={n,u=\tgmESE},draw=Green],
  fw -- [going=ne]
  "Cragne\\Family\\Plot" [x=1,y=1,exit=nw,draw=Green] -- [going=in]
  "Crypt" [x=2,y=1,draw=Green]
};
\end{gamemap}
\end{adjustbox}

\vspace*{\stretch{1}}
\newpage\phantomsection\label{sec:route-7}
\begin{description}
\item[{\hyperref[sec:req-Town-Square]{Town Square}}] ~\\
  \emph{Go SW}
\item[{\hyperref[sec:req-Drinking-Fountain]{Drinking Fountain}}] ~\\
  \emph{Go NE, W}
\item[{\hyperref[sec:req-Church-Exterior]{Church Exterior}}] ~\\
  \emph{Go in}
\item[{\hyperref[sec:req-Church-Lobby-Space]{Church Lobby-Space}}] ~\\
  \emph{Go up}
\item[{\hyperref[sec:req-Steeple-0]{Steeple}}] (partial)\\
  \emph{Go down, W}
\item[{\hyperref[sec:req-Chapel]{Chapel}}] ~\\
  \emph{Go S, down}
\item[{\hyperref[sec:req-Church-Basement]{Church Basement}}] ~\\
  \emph{Go up, N, E out -- Go E, E, E, E, NE, N* -- Go N, N, N, N }
\end{description}

\vspace*{\stretch{1}}
\begin{adjustbox}{max width=\textwidth,center}
\sffamily\footnotesize
\begin{gamemap}
\graph [no placement] {
  "River\\Walk" [x=2,y=-5,exit=n,tunnel,draw=Green] -- [going=nw]
  ts/"Town\\Square" [x=1,y=-4,entry,draw=Green] -- [going=w]
  cx/"Church\\Exterior" [x=0,y=-4,exit=n,tunnel,draw=Green] -- [go={in=\tgmWSW}{o=\tgmENE}]
  cn/"Church\\Lobby-\\Space" [x=-1,y=-4,draw=Green] -- [going=u]
  "Steeple" [x=-1,y=-3,draw=DarkOrange],
  cn -- [going=w]
  "Chapel" [x=-2,y=-4,draw=Green] -- [going=s]
  co/"Church\\Office" [x=-2,y=-5,draw=Green] -- [going=d]
  "Church\\Basement" [x=-2,y=-6,exit=d,draw=Green],
  ol/"Outside\\the Library" [x=1,y=-3,exits={nw,e,w},tunnel,draw=Green] -- [going=s]
  ts -- [going=sw]
  "Drinking\\Fountain" [x=0,y=-5,draw=Green]
};
\end{gamemap}
\end{adjustbox}

\vspace*{\stretch{1}}
\newpage\phantomsection\label{sec:route-8}
\begin{description}
\item[{\hyperref[sec:req-Dining-Room]{Dining Room}}] (optional)\\
  \emph{E, W, W}
\item[{\hyperref[sec:req-Kitchen]{Kitchen}}] ~\\
  \emph{Go down}
\item[{\hyperref[sec:req-Basement]{Basement}}] ~\\
  \emph{Go N}
\end{description}
\vskip-\dimexpr\parskip + 2\partopsep + \smallskipamount\relax
\begin{multicols}{2}
\begin{description}
\item[{\hyperref[sec:req-Cold-Storage]{Cold Storage}}] ~\\
  \emph{Go S, E}
\item[{\hyperref[sec:req-Pantry]{Pantry}}] ~\\
  \emph{Go E}
\item[{\hyperref[sec:req-Workroom]{Workroom}}] ~\\
  \emph{Go W, W, W}
\item[{\hyperref[sec:req-Wine-Cellar]{Wine Cellar}}] ~\\
  \emph{Go W}
\item[{\hyperref[sec:req-Laboratory]{Laboratory}}] ~\\
  \emph{Go E, E, S}
\item[{\hyperref[sec:req-Boiler-Room]{Boiler Room}}] ~\\
  \emph{Go down}
\end{description}
\end{multicols}
\vskip-\dimexpr\parskip + 2\partopsep\relax
\begin{adjustbox}{minipage=\dimexpr.5\linewidth - .5\multicolsep\relax,right}
\setlength{\parskip}{\currentparskip}%
\begin{description}
\item[{\hyperref[sec:req-Malign-Tunnel]{Malign Tunnel}}] (optional)\\
  \emph{Go SW, up}
\item[{\hyperref[sec:req-Courtyard]{Courtyard}}] ~\\
  \emph{Go N -- Go NW, N* --\\Go N, N, N, up, E, S}
\end{description}
\end{adjustbox}

\vspace*{\stretch{1}}
\begin{adjustbox}{max width=\textwidth}
\sffamily\footnotesize
\begin{gamemap}[set grid={9em}{5em}]
\graph [no placement] {
  fw/"Front\\Walk" [x=0,y=1.5,entry,exits={nw,ne,s},draw=Green] -- [going=n,dashed]
  lb/"Landing at\\the Bottom\\of Stairs" [x=0,y=3,exit={u=\tgmESE},draw=Green] -- [going=n]
  dr/"Dining\\Room" [x=0,y=5,exit=e,draw=Green] -- [going=w]
  "Kitchen" [x=-1,y=5,draw=Green] -- [go={d}{u=\tgmNE}]
  b/"Basement" [x=-2,y=4,draw=Green] -- [going=n]
  "Cold Storage\\Room" [x=-2,y=5,draw=Green],
  b -- [going=s]
  "Boiler\\Room" [x=-2,y=3,draw=Green] -- [going=d]
  "Malign\\Tunnel" [x=-2,y=2,tunnel,draw=Green] -- [going=sw]
  at/"Amorphous\\Tunnel" [x=-2.8,y=1,exit=w] -- [go={u=\tgmNNW}{d=\tgmSSE}]
  "Courtyard" [x=-3.6,y=2,exits={e,n},draw=Green],
  b -- [going=e]
  "Pantry" [x=-1.2,y=4,draw=Green] -- [going=e]
  "Workroom" [x=-0.4,y=4,draw=Green],
  b -- [going=w]
  "Wine\\Cellar" [x=-2.8,y=4,draw=Green] -- [going=w]
  "Laboratory" [x=-3.6,y=4,draw=Green],
  dr -- [going=e]
  "Sitting\\Room" [x=0.8,y=5],
};
\end{gamemap}
\end{adjustbox}

\vspace*{\stretch{1}}
\newpage\phantomsection\label{sec:route-9}
\begin{description}
\item[{\hyperref[sec:req-Hallway-South-1]{Hallway South}}] ~\\
  \emph{Go S}
\item[{\hyperref[sec:req-Balcony]{Balcony}}] ~\\
  \emph{Go N, N}
\item[{\hyperref[sec:req-Upstairs-Hall-N]{Upstairs Hall, North End}}] ~\\
  \emph{Go E}
\item[{\hyperref[sec:req-Master-Bedroom]{Master Bedroom}}] ~\\
  \emph{Go E}
\item[{\hyperref[sec:req-Shadowy-Closet]{Shadowy Closet}}] ~\\
  \emph{Go up}
\end{description}

\vspace*{\stretch{1}}
\begin{adjustbox}{center}
\sffamily\footnotesize
\begin{gamemap}
\graph [no placement] {
  { [x=1]
  fw/"Front\\Walk" [y=1,entry,exits={nw,ne,s},draw=Green] -- [going=n]
  "Mudroom" [y=2,draw=Green] -- [going=n]
  f/"Foyer" [y=3,draw=Green,exits={e,w}] -- [going=n]
  "Landing at\\the Bottom\\of Stairs" [y=4,exit=n,draw=Green]
  } -- [go={u=\tgmENE}{d=\tgmWSW}]
  "Top of\\Stairs" [x=2,y=5,draw=Green] -- [going=e]
  hn/"Upstairs\\Hall N" [x=3,y=5,draw=Green] -- [going=s]
  hs/"Hallway\\South" [x=3,y=4,draw=Green] -- [going=e]
  "Library" [x=4,y=4,draw=Green],
  "Study" [x=2,y=4,draw=Green] -- [going=e]
  hs -- [going=s]
  "Balcony" [x=3,y=3,tunnel,draw=Green],
  hn -- [going=n]
  "Carol's Room" [x=3,y=6,draw=Green],
  hn -- [going=e]
  "Master\\Bedroom" [x=4,y=5,draw=Green] -- [going=e]
  "Shadowy\\Closet" [x=5,y=5,exit=u,draw=Green]
};
\end{gamemap}
\end{adjustbox}

\vspace*{\stretch{1}}
\newpage\phantomsection\label{sec:route-10}
\begin{description}
\item[{\hyperref[sec:req-Attic]{Attic}}] ~\\
  \emph{Go S}
\item[{\hyperref[sec:req-Disheveled-Studio]{Disheveled Studio}}] ~\\
  \emph{Go SE}
\item[{\hyperref[sec:req-Abandoned-Nursery-0]{Abandoned Nursery}}] (establish requirements)\\
  \emph{Go N}
\item[{\hyperref[sec:req-Invasive-Library]{Invasive Library}}] ~\\
  \emph{Go SE, W}
\item[{\hyperref[sec:req-Science-Tower]{Science Tower}}] ~\\
  \emph{Go E (or SW, N, SE for completeness)}
\item[{\hyperref[sec:req-Branching-Corridor]{Branching Corridor}}] ~\\
  \emph{Go SE, NW, E -- Go down, W, W, W, down, S, S, S* -- Go NW, NE, in}
\end{description}

\vspace*{\stretch{1}}
\begin{adjustbox}{center}
\sffamily\footnotesize
\begin{gamemap}[set grid={9em}{5em}]
\graph [no placement] {
  "Shadowy\\Closet" [x=5,y=2,entry,exit=w,draw=Green] -- [go={u=\tgmSSW}{d=\tgmNNE}]
  a/"Attic" [x=5,y=1,draw=Green,tunnel] -- [going=w]
  bc/"Branching\\Corridor" [x=4,y=1,draw=Green] -- [going=w]
  "Science\\Tower" [x=3,y=1,draw=Green] -- [going=sw]
  an/"Abandoned\\Nursery" [x=2,y=0] -- [going=n]
  "Invasive\\Library" [x=2,y=2,draw=Green] -- [going=se]
  bc -- [going=se]
  "Observatory" [x=5,y=0],
  a -- [rounded corners,tgmedge,to path={ -- (5,0.5) node [transit node,pos=0] {S} -- (2.25,0.5) -- (1.75,1.5) -- (1,1.5) -- (\tikztotarget) node [transit node,pos=1] {N} }]
  "Disheveled\\Studio" [x=1,y=1,draw=Green] -- [going=se]
  an,
};
\end{gamemap}
\end{adjustbox}

\vspace*{\stretch{1}}
\newpage\phantomsection\label{sec:route-11}
\begin{description}
\item[{\hyperref[sec:req-Shambolic-Shack]{Shambolic Shack}}] ~\\
  \emph{Go out, SW, SE, N, N, W, N}
\item[{\hyperref[sec:req-Rec-Room]{Rec Room}}] ~\\
  \emph{Go S, W, S, S -- Go S, SE* -- Go in}
\item[{\hyperref[sec:req-The-Meatpacking-Plant]{The Meatpacking Plant}}] ~\\
  \emph{Go out -- Go NW, SW* -- Go S, E}
\end{description}

\vspace*{\stretch{1}}
\begin{adjustbox}{max width=\textwidth,center}
\sffamily\footnotesize
\begin{gamemap}
\graph[no placement] {
  "Constabulary\\Road" [x=7,y=-5,exits={n,w,s},tunnel,draw=Green] -- [going=ne]
  hp/"Hillside\\Path" [x=8,y=-4,exit=sw,draw=Green] -- [going=se]
  "Outside\\the Plant" [x=9,y=-5,tunnel,draw=Green] -- [going=in]
  mp/"The\\Meatpacking\\Plant" [x=10.5,y=-6,draw=Green] -- [going=u]
  "Dusty\\Office" [x=10.5,y=-5,draw=Green],
  mp -- [going=w]
  "Bathroom of\\the Meatpacking\\Plant" [x=9,y=-6,draw=Green],
  hp -- [going=n]
  fw/"Front\\Walk" [x=8,y=-3,tunnel,draw=Green] -- [going=nw]
  og/"Outside the\\Greenhouse" [x=6,y=-1,draw=Green] -- [going=ne]
  bg/"The Cragne\\Manor's Back\\Garden" [x=8,y=1,exit=se,draw=Green],
  og -- [go={in=\tgmWSW}{o=\tgmENE}],
  bg -- [going=in]
  "The\\Shambolic\\Shack" [x=9.5,y=1,entry,draw=Green],
  fw/"Front\\Walk" [x=8,y=0,tunnel,exit=ne,draw=Green] -- [going=n]
  "Mudroom" [x=8,y=-2,draw=Green] -- [going=n]
  f/"Foyer" [x=8,y=-1,draw=Green,exits={n,w}] -- [going=e]
  "Gallery" [x=9,y=-1,draw=Green] -- [going=n]
  "Rec\\Room" [x=9,y=0,draw=Green] -- [going=e]
  "Music\\Room" [x=10,y=0,draw=Green],
};
\end{gamemap}
\end{adjustbox}

\vspace*{\stretch{1}}
\newpage\phantomsection\label{sec:route-12}
\begin{description}
\item[{\hyperref[sec:req-Curiosity-Shop]{Curiosity Shop}}] ~\\
  \emph{Go W, down}
\item[{\hyperref[sec:req-Amorphous-Tunnel]{Amorphous Tunnel}}] ~\\
  \emph{Go W}
\item[{\hyperref[sec:req-Narrow-Straits]{Narrow Straits}}] ~\\
  \emph{Go W}
\item[{\hyperref[sec:req-Subterranean-Tunnel]{Subterranean Tunnel}}] (optional)\\
  \emph{Go up, S -- Go NW, N* -- E}
\item[{\hyperref[sec:req-Backwater-Public-Library]{Backwater Public Library}}] ~\\
  \emph{Go W -- Go S, E, E* -- Go N}
\item[{\hyperref[sec:req-The-Invisible-Worm]{The Invisible Worm}}] ~\\
  \emph{Go S, E -- Go NE, N* -- Go NW, in}
\end{description}

\vspace*{\stretch{1}}
\begin{adjustbox}{max width=\textwidth,center}
\sffamily\footnotesize
\begin{gamemap}[set grid={9em}{6em}]
\graph [no placement] {
  ub/"Under the\\Bridge" [x=5,y=-5.25,exit={d=\tgmSE},draw=Green] -- [going=s]
  "River\\Walk" [x=5,y=-6,tunnel,draw=Green] -- [going=nw]
  ts/"Town\\Square" [x=4,y=-5,exits={sw,w},draw=Green],
  "Backwater\\Public\\Library" [x=5,y=-4,draw=Green] -- [going=w]
  ol/"Outside\\the Library" [x=4,y=-4,exits={nw,w},tunnel,draw=Green] -- [going=s]
  ts -- [going=e]
  "Bridge" [x=5,y=-4.7,draw=Green] -- [going=e]
  op/"Outside\\Pub" [x=6,y=-5,exit=w,draw=Green] -- [going=e]
  "Constabulary\\Road" [x=7,y=-5,entry,exits={n,ne},draw=Green] -- [going=s]
  ct/"Courtyard" [x=7,y=-6,draw=Green] -- [going=e]
  "Curiosity\\Shop" [x=8,y=-6,draw=Green],
  op -- [going=n]
  "The\\Invisible\\Worm" [x=6,y=-4,draw=Green],
  ct -- [going=d]
  { [y=-7]
  at/"Amorphous\\Tunnel" [x=7,exit=ne,draw=Green] -- [going=w]
  "Narrow\\Straits" [x=6,draw=Green] -- [going=w]
  "Subterranean\\Tunnel" [x=5,exit=nw,draw=Green] -- [go={u=\tgmNE}{d=\tgmSE}]
  ub
  }
};
\end{gamemap}
\end{adjustbox}

\vspace*{\stretch{1}}
\newpage\phantomsection\label{sec:route-13}
\begin{description}
\item[{\hyperref[sec:req-Greenhouse]{Greenhouse}}] ~\\
  \emph{Go out, SE, N, N, W}
\item[{\hyperref[sec:req-Court]{Court}}] ~\\
  \emph{Go E -- Go N, up, E, E, E, up* -- Go S, SE}
\item[{\hyperref[sec:req-Abandoned-Nursery]{Abandoned Nursery}}] ~\\
  \emph{Go NW, N -- Go down, W, W, W, down* -- Go N, E}
\item[{\hyperref[sec:req-Sitting-Room]{Sitting Room}}] ~\\
  \emph{Go W, S, S, S, S -- Go S, SW, W, W, W, W* -- Go in, up}
\item[{\hyperref[sec:req-Steeple]{Steeple}}] ~\\
  \emph{Go down, out -- Go N, N, N*}
\item[{\hyperref[sec:req-Train-Station-Lobby]{Train Station Lobby}}] ~\\
  \emph{Go E}
\item[{\hyperref[sec:req-Station-Security-Room]{Station Security Room}}] ~\\
  \emph{Go W -- Go S, S, S, NE, NE* -- Go SE}
\item[{\hyperref[sec:req-Shack-Exterior]{Shack Exterior}}] (door)\\
  \emph{Go in}
\item[{\hyperref[sec:req-Inside-the-Shack]{Inside the Shack}}] ~\\
  \emph{Go out, SE -- Go S, E, E, E, NE, N, N, N, N, up, E, E, E, up* -- Go E, SE}
\item[{\hyperref[sec:req-Observatory]{Observatory}}]
\end{description}

\vspace*{\stretch{1}}
\begin{center}
\emph{See the full game maps in \cref{sec:full-map}.}
\end{center}

\vspace*{\stretch{1}}
\newpage
\section{Room requirements and objectives}\label{sec:sum}

\subsection{Railway Platform}\label{sec:req-Railway-Platform-0}

You will find a locked item and a trapped object.
Now you have one or two things to look out for on your travels.

You don't need to take any of the items here with you (yet).

\hyperref[sec:route]{\emph{Return to route}}

\newpage
\subsection{Train Station Lobby}\label{sec:req-Train-Station-Lobby-0}

At this point, all you need is the coffee.

You'll need to find a key for the green door elsewhere.

\hyperref[sec:route]{\emph{Return to route}}

\newpage
\subsection{Train Station Restroom}\label{sec:req-Train-Station-Restroom}

There's an object in here that you need.

For the solution to this room, see \cref{sec:sol-Train-Station-Restroom}.

\hyperref[sec:route]{\emph{Return to route}}

\newpage
\subsection{Exterior of Train Station}\label{sec:req-Exterior-of-Train-Station}

There is one item here you should take with you --
for further information about it, see \cref{sec:nb-doll} --
and another one that you should use and can then drop.

You need to get past that woman somehow\xelip

For the solution to this room, see \cref{sec:sol-Exterior-of-Train-Station}.

\hyperref[sec:route]{\emph{Return to route}}

\newpage
\subsection{Milkweed}\label{sec:req-Milkweed}
Hidden in this area is key piece of information.

There is indeed a lot of information here.
If you would like to know what sort of thing you need to look out for,
see the note in \cref{sec:nb-Variegated-Court}.

For the solution to this room, see \cref{sec:sol-Milkweed}.

\hyperref[sec:route]{\emph{Return to route}}

\newpage
\subsection{The Churchyard}\label{sec:req-Churchyard}

Your mission here is to gain entry into the mausoleum.

There is also an item here that you need to take with you.

Do not be alarmed by the fact that you silently acquire aviator goggles
on entering the location and lose them again on leaving. This sort of thing
happens.

For the solution to this room, see \cref{sec:sol-Churchyard}.

\hyperref[sec:route]{\emph{Return to route}}

\newpage
\subsection{Mausoleum}\label{sec:req-Mausoleum}

Take the book away with you.

For the solution to this room, see \cref{sec:sol-Mausoleum}.

\hyperref[sec:route]{\emph{Return to route}}

\newpage
\subsection{The Dim Recesses of the Forest}\label{sec:req-The-Dim-Recesses-of-the-Forest}

You need to learn a key piece of information here.

For the solution to this room, see \cref{sec:sol-The-Dim-Recesses-of-the-Forest}.

\hyperref[sec:route-1]{\emph{Return to route}}

\newpage
\subsection{Shack Exterior}\label{sec:req-Shack-Exterior-0}

You need to obtain an item from this room.

For the solution of how to do this, see \cref{sec:sol-Shack-Exterior-0}.

You cannot gain entry to the shack at this point.

\hyperref[sec:route-1]{\emph{Return to route}}

\newpage
\subsection{Outside the Library}\label{sec:req-Outside-the-Library}

Read the contents of the noticeboard.

\hyperref[sec:route-1]{\emph{Return to route}}

\newpage
\subsection{Backwater Public Library}\label{sec:req-Public-Library-0}

There is an item here you will need, but you can't get it yet.

You can, however, get a very useful \enquote{shopping list} of items you need
to collect before returning.

Even if you have some of those items already, you may as well hang on to them
for now.

For more details, see the steps in \cref{sec:sol-Public-Library-0}
and the note in \cref{sec:nb-library-books}.

\hyperref[sec:route-1]{\emph{Return to route}}

\newpage
\subsection{Estate Agent's Office}\label{sec:req-Estate-Agent's-Office}

The introduction to this room mentions a couple of items you didn't have before.
Check your inventory and examine them before doing anything else.
(This is not strictly necessary but it's a nice touch.)
There's a note about one of the items in \cref{sec:nb-backpack}.

I don't need to tell you what else to do here as the narrative does a very good job of that itself.

When you have solved the room and thoroughly examined the stolen (reclaimed?)
\emph{something} -- you'll know what I'm getting at when you do it --
there's a note about it in \cref{sec:nb-cyphered-message}.

For the solution to this room, see \cref{sec:sol-Estate-Agent's-Office}.
If you just want to know where to start, there's a hint at the bottom of this page.

\hyperref[sec:route-1]{\emph{Return to route}}

\vfill
You'd heard about one property in the game before you even started playing.

\newpage
\subsection{Town Square}\label{sec:req-Town-Square-0}

For interest, have a chat with the man by the bridge.

For a visualisation of the emblem, see the note in \cref{sec:nb-town-square-emblem}.
You will pick up a clue about what to do with this later on.

\hyperref[sec:route-1]{\emph{Return to route}}

\newpage
\subsection{River Walk}\label{sec:req-River-Walk}

There is an item hidden here that you need for another location.

You don't need anything from anywhere else,
but you may get a more obvious hint if you have something sharp on you.
If you haven't come across one, you could skip ahead to the Bridge
and come back here after completing Backwater Jail.

For the solution to this room, see \cref{sec:sol-River-Walk}.

\hyperref[sec:route-1]{\emph{Return to route}}

\newpage
\subsection{Under the Bridge}\label{sec:req-Under-the-Bridge-0}

You need to learn a key piece of information here.

For the solution, see \cref{sec:sol-Under-the-Bridge-0}.

You cannot open the hatch at this point.

\hyperref[sec:route-1]{\emph{Return to route}}

\newpage
\subsection{Railway Platform}\label{sec:req-Railway-Platform}

You need the ID card from River Walk.

You should now be able to get the item out of the vending machine.

For the solution to this room, see \cref{sec:sol-Railway-Platform}.

\hyperref[sec:route-2]{\emph{Return to route}}

\newpage
\subsection{The Old Well}\label{sec:req-The-Old-Well}

You need the golden eyepiece from the Railway Platform vending machine.

Your aim here is to open the well.

You will probably need to make notes to solve this room,
and make liberal use of \textsc{undo}.
There are many ways to end the game here, and they are worth reading.

For the solution to this room, see \cref{sec:sol-The-Old-Well}.

\hyperref[sec:route-2]{\emph{Return to route}}

\newpage
\subsection{Circular Room}\label{sec:req-Circular-Room}

If you were exhausted permuting your way through the previous room,
this one may come as something of a relief.

Examine the stones and take the cash.

\hyperref[sec:route-2]{\emph{Return to route}}

\newpage
\subsection{Bridge}\label{sec:req-Bridge}

You need the flashlight from the Churchyard.

Your aim is simply (!) to get across the bridge.

For the solution to this room, see \cref{sec:sol-Bridge}.

\hyperref[sec:route-2]{\emph{Return to route}}

It is rooms like this (and the Circular Room) that make you appreciate
how astonishingly consistent the rest of the game feels.

\newpage
\subsection{Outside Pub}\label{sec:req-Outside-Pub}

There is vital information in the newspaper,
but how can you get one (using only what's already here)?

For the solution to this room, see \cref{sec:sol-Outside-Pub}.

\hyperref[sec:route-2]{\emph{Return to route}}

\newpage
\subsection{Constabulary Road}\label{sec:req-Constabulary-Road}

There is nothing in particular to solve here, but I recommend examining
everything and talking to the man. If you manage to end the game here and
find the ending a bit odd, I think that's because it has leaked in from another
room.

The north gate is closed but unlocked.

If you think you ought to be able to unblock the passage south,
hold that thought: you can't do anything about it at the moment.

\hyperref[sec:route-2]{\emph{Return to route}}

\newpage
\subsection{Backwater Jail}\label{sec:req-Backwater-Jail}

There is a library book here you need to liberate.

For the solution to this room, see \cref{sec:sol-Backwater-Jail}.

\hyperref[sec:route-2]{\emph{Return to route}}

\newpage
\subsection{Padded Cell}\label{sec:req-Padded-Cell}

A classic escape room scenario!

While you are here, make sure you discover a clue about another Alderman.

For the solution to this room, see \cref{sec:sol-Padded-Cell}.

\hyperref[sec:route-2]{\emph{Return to route}}

\newpage
\subsection{Outside the Plant}\label{sec:req-Outside-the-Plant}

You need to gain entry to the meatpacking plant.

For the solution to this room, see \cref{sec:sol-Outside-the-Plant}.

\hyperref[sec:route-3]{\emph{Return to route}}

\newpage
\subsection{Dusty Office}\label{sec:req-Dusty-Office}

There are clues about another Alderman here.

For the solution to this room, see \cref{sec:sol-Dusty-Office}.

\hyperref[sec:route-3]{\emph{Return to route}}

\newpage
\subsection{Bathroom of the Meatpacking Plant}\label{sec:req-Bathroom-of-the-Meatpacking-Plant}

There is information about another Alderman here.

For the solution to this room, see \cref{sec:sol-Bathroom-of-the-Meatpacking-Plant}.

\hyperref[sec:route-3]{\emph{Return to route}}

\newpage
\subsection{Hillside Path}\label{sec:req-Hillside-Path}

The woman can give you an idea of what to expect in one of the rooms in the Manor.

\hyperref[sec:route-3]{\emph{Return to route}}

\newpage
\subsection{Front Walk}\label{sec:req-Front-Walk}

There's no problem to solve; just enjoy the descriptions.

\hyperref[sec:route-3]{\emph{Return to route}}

\newpage
\subsection{Cragne Family Plot}\label{sec:req-Cragne-Family-Plot}

Believe it or not, there is a library book hidden somewhere here.

For the solution to this room, see \cref{sec:sol-Cragne-Family-Plot}.

You cannot open the crypt door at this point.

\hyperref[sec:route-3]{\emph{Return to route}}

\newpage
\subsection{Back Garden}\label{sec:req-Back-Garden}

In amongst the vines is a shack; you need to find your way into it.

For the solution to this room, see \cref{sec:sol-Back-Garden}.

\hyperref[sec:route-3]{\emph{Return to route}}

\newpage
\subsection{Shambolic Shack}\label{sec:req-Shambolic-Shack-0}

There is a key to find here.

There is something else as well but you won't be able to get at it yet.

For the solution, see \cref{sec:sol-Shambolic-Shack-0}.

\hyperref[sec:route-3]{\emph{Return to route}}

\newpage
\subsection{Under the Bridge}\label{sec:req-Under-the-Bridge}

Unlock the hatch.
If you're not sure which key to use, see \cref{sec:sol-Under-the-Bridge}.

Now is a great time to tidy up your inventory.

\hyperref[sec:route-4]{\emph{Return to route}}

\newpage
\subsection{Tunnel Entrance}\label{sec:req-Tunnel-Entrance}

There are two linked puzzles here. One is to sort out the lighting and the
other is to open a secret door.

For the solution to this room, see \cref{sec:sol-Tunnel-Entrance}.

\hyperref[sec:route-5]{\emph{Return to route}}

\newpage
\subsection{Small Chamber}\label{sec:req-Small-Chamber}

There is a library book hidden here.

For the solution to this room, see \cref{sec:sol-Small-Chamber}.
It is a bit finicky so I have concentrated on the bits that might trip you up,
without revealing the exact answer to the puzzle.

\hyperref[sec:route-5]{\emph{Return to route}}

\newpage
\subsection{Tiny Windowless Office}\label{sec:req-Tiny-Windowless-Office}

There is a key to find and also a clue about another Alderman.

You cannot unlock the north door at this point.

For the solution to this room, see \cref{sec:sol-Tiny-Windowless-Office}.

\hyperref[sec:route-5]{\emph{Return to route}}

\newpage
\subsection{Mudroom}\label{sec:req-Mudroom}

There is an item here you need to take with you.

There is also a door to unlock, for which you will need a key from another room.

For the solution to this room, see \cref{sec:sol-Mudroom}.

\hyperref[sec:route-5]{\emph{Return to route}}

\newpage
\subsection{Foyer}\label{sec:req-Foyer}

There is a hidden item in this room, but I never found a use for it.
Perhaps you can discover one?

\hyperref[sec:route-5]{\emph{Return to route}}

\newpage
\subsection{Landing at the Bottom of Stairs}\label{sec:req-Landing-at-the-Bottom-of-Stairs}

There is a library book hidden here.

For the solution to this room, see \cref{sec:sol-Landing-at-the-Bottom-of-Stairs}.

\hyperref[sec:route-5]{\emph{Return to route}}

\newpage
\subsection{Top of Stairs}\label{sec:req-Top-of-Stairs}

There is a library book hidden here.

For the solution to this room, see \cref{sec:sol-Top-of-Stairs}.

\hyperref[sec:route-5]{\emph{Return to route}}

\newpage
\subsection{Nursery -- Hillside Path -- Carol's Room}\label{sec:req-Nursery--Hillside-Path--Carol's-Room}

You need to be carrying the teapot from the Mudroom to enter the Nursery.

You will, eventually, get a vital clue from here,
but there's a bit of to-ing and fro-ing to do as you resolve the relationship
between Christabell and Carol.

For the solution to this room, see \cref{sec:sol-Nursery--Hillside-Path--Carol's-Room}.

\hyperref[sec:route-5]{\emph{Return to route}}

\newpage
\subsection{Hallway South}\label{sec:req-Hallway-South}

There is a library book in this location.

I did not find a use for the remaining item(s).

For the solution, see \cref{sec:sol-Hallway-South}.

You will need to find the key for the south door somewhere else.

\hyperref[sec:route-5]{\emph{Return to route}}

\newpage
\subsection{Library}\label{sec:req-Library}

One of the books here belongs to the public library rather than this one.
(It is not one of the obvious ones.)

For the solution to this room, see \cref{sec:sol-Library}.

\hyperref[sec:route-5]{\emph{Return to route}}

\newpage
\subsection{Study}\label{sec:req-Study}

While there are some evocative descriptions here,
it is true that the key is all you need.

\hyperref[sec:route-5]{\emph{Return to route}}

\newpage
\subsection{Gallery}\label{sec:req-Gallery}

You need to unblock the passage north.

For the solution to this room, see \cref{sec:sol-Gallery}.

\hyperref[sec:route-6]{\emph{Return to route}}

\newpage
\subsection{Music Room}\label{sec:req-Music-Room}

You can learn about another Alderman here.

Once you have established what the podium does, you may find the
diagram in \cref{sec:nb-Music-Room} helpful.

For the solution to this room, see \cref{sec:sol-Music-Room}.

\hyperref[sec:route-6]{\emph{Return to route}}

\newpage
\subsection{Court}\label{sec:req-Court-0}

At this point, you just need to take the key.

If you're interested, here's what the colours mean:

\begin{itemize}
\item Xanthic: tending towards yellow
\item Mazarine: deep purplish blue
\item Rufous: reddish brown
\item Croceate: saffron yellow
\item Icterine: yellowish, or marked with yellow
\item Puce: dark red to purplish brown
\item Fulvous: tawny; reddish, brownish yellow
\item Fuscous: dusky brownish grey
\item Niveous: snowy white
\item Cesious:  bluish grey
\item Griseous: white mottled with black or brown
\item Eburnean: ivory
\end{itemize}

\hyperref[sec:route-6]{\emph{Return to route}}

\newpage
\subsection{Cragne Family Plot}\label{sec:req-Cragne-Family-Plot-1}

You need a key from another location to open the Crypt.

If you are not sure which one, see \cref{sec:sol-Cragne-Family-Plot-1}.

\hyperref[sec:route-6]{\emph{Return to route}}

\newpage
\subsection{Crypt}\label{sec:req-Crypt}

You can learn about another Alderman here.

For the solution to this room, see \cref{sec:sol-Crypt}.

\hyperref[sec:route-6]{\emph{Return to route}}

\newpage
\subsection{Town Square}\label{sec:req-Town-Square}

You need to have completed the Nursery\slash Hillside Path\slash Carol's Room,
and obtained a \hyperref[sec:nb-picture-riddle]{worn out, decaying picture}
before tackling this room.

There is an item hidden here that contains a clue you need for another location.

For the solution to this room, see \cref{sec:sol-Town-Square}.

\hyperref[sec:route-7]{\emph{Return to route}}

\newpage
\subsection{Drinking Fountain}\label{sec:req-Drinking-Fountain}

You need to have read 9 library books before you're able to tackle this area.
You'll be able to tell if you've done this because a well dressed man will be here.

There is another library book hidden in this location.

For the solution to this room, see \cref{sec:sol-Drinking-Fountain}.

\hyperref[sec:route-7]{\emph{Return to route}}

\newpage
\subsection{Church Exterior}\label{sec:req-Church-Exterior}

You need a key from another location to open the Church door. Keep hold of it.

If you are not sure which key that is, see \cref{sec:sol-Church-Exterior}.

\hyperref[sec:route-7]{\emph{Return to route}}

\newpage
\subsection{Church Lobby-Space (Narthex)}\label{sec:req-Church-Lobby-Space}

There is a library book hidden here.

For the solution to this room, see \cref{sec:sol-Church-Lobby-Space}.

\hyperref[sec:route-7]{\emph{Return to route}}

\newpage
\subsection{Steeple}\label{sec:req-Steeple-0}

All you need to do at this point is retrieve the key.

\hyperref[sec:route-7]{\emph{Return to route}}

\newpage
\subsection{Chapel}\label{sec:req-Chapel}

You need a key from another location to open the south door.

You will also need an item from an earlier room to the release the object that is hidden here.
That object will come in handy a bit later on.

For the solution to this room, see \cref{sec:sol-Chapel}.

\hyperref[sec:route-7]{\emph{Return to route}}

\newpage
\subsection{Church Basement}\label{sec:req-Church-Basement}

You need the clue from the Town Square in order to discover an important message.

For the solution to this room, see \cref{sec:sol-Church-Basement}.

\hyperref[sec:route-7]{\emph{Return to route}}

\newpage
\subsection{Dining Room}\label{sec:req-Dining-Room}

There is nothing you particularly need from this room,
but there's something interesting to do.

For the solution to this room, see \cref{sec:sol-Dining-Room}.

\hyperref[sec:route-8]{\emph{Return to route}}

\newpage
\subsection{Kitchen}\label{sec:req-Kitchen}

You need a key from another location to open the trap door.

If you are not sure which one, see \cref{sec:sol-Kitchen}.

\hyperref[sec:route-8]{\emph{Return to route}}

\newpage
\subsection{Basement}\label{sec:req-Basement}

Believe it or not, there are lots of books hidden here,
one of which you need to take away with you.

For the solution to this room, see \cref{sec:sol-Basement}.

\hyperref[sec:route-8]{\emph{Return to route}}

\newpage
\subsection{Cold Storage}\label{sec:req-Cold-Storage}

You can learn about another Alderman here.

For the solution to this room, see \cref{sec:sol-Cold-Storage}.

\hyperref[sec:route-8]{\emph{Return to route}}

\newpage
\subsection{Pantry}\label{sec:req-Pantry}

There is a special \hyperref[sec:nb-Church-Basement]{recorded message}
you need to have heard before tackling this room.

It will help you discover a clue you'll need for another room.
You will know it when you see it because it rhymes.

For the solution to this room, see \cref{sec:sol-Pantry}.

\hyperref[sec:route-8]{\emph{Return to route}}

\newpage
\subsection{Workroom}\label{sec:req-Workroom}

By the time you have finished, there will be a library book hidden here.

For the solution to this room, see \cref{sec:sol-Workroom}.

\hyperref[sec:route-8]{\emph{Return to route}}

\newpage
\subsection{Wine Cellar}\label{sec:req-Wine-Cellar}

There is an object hidden here that you need,
and you also need to unblock the west exit.

For the solution to this room, see \cref{sec:sol-Wine-Cellar}.

\hyperref[sec:route-8]{\emph{Return to route}}

\newpage
\subsection{Laboratory}\label{sec:req-Laboratory}

You need an object from another room in order to resolve this room
satisfactorily, but I can't tell you which one without giving the game away.

There is an object here that you need.

For the solution to this room, see \cref{sec:sol-Laboratory}.

\hyperref[sec:route-8]{\emph{Return to route}}

\newpage
\subsection{Boiler Room}\label{sec:req-Boiler-Room}

There is a library book hidden here.

For the solution to this room, see \cref{sec:sol-Boiler-Room}.

\hyperref[sec:route-8]{\emph{Return to route}}

\newpage
\subsection{Malign Tunnel}\label{sec:req-Malign-Tunnel}

There's nothing to do here except enjoy the descriptions.

\hyperref[sec:route-8]{\emph{Return to route}}

\newpage
\subsection{Courtyard}\label{sec:req-Courtyard}

Unblock the exit north. You don't really need to do this but it's convenient
and ties up a loose end.

For the solution to this room, see \cref{sec:sol-Courtyard}.

\hyperref[sec:route-8]{\emph{Return to route}}

\newpage
\subsection{Hallway South}\label{sec:req-Hallway-South-1}

You need a key from another location to open the south door.

If you are not sure which one, see \cref{sec:sol-Hallway-South-1}.

\hyperref[sec:route-9]{\emph{Return to route}}

\newpage
\subsection{Balcony}\label{sec:req-Balcony}

There is an object hidden here that you need.

For the solution to this room, see \cref{sec:sol-Balcony}.

\hyperref[sec:route-9]{\emph{Return to route}}

\newpage
\subsection{Upstairs Hall, North End}\label{sec:req-Upstairs-Hall-N}

This room goes through three distinct states.
In the initial state you can't do anything.

In the second state, you can discover clues about another Alderman.

In the third state, you can resolve the matter of the northeast door.

You need a key from elsewhere to unlock the east door.

For the solution to this room, see \cref{sec:sol-Upstairs-Hall-N}.

\hyperref[sec:route-5]{\emph{Return to route, having established requirements}}

\hyperref[sec:route-9]{\emph{Return to route, having unlocked the east door}}

\newpage
\subsection{Master Bedroom}\label{sec:req-Master-Bedroom}

You can, if you like, just pass through this room,
but where would be the fun in that?

As always,
unless you actually end the game here there is no reason to \textsc{undo}.

For the solution to this room, see \cref{sec:sol-Master-Bedroom}.

\hyperref[sec:route-9]{\emph{Return to route}}

\newpage
\subsection{Shadowy Closet}\label{sec:req-Shadowy-Closet}

You need an item from another area to solve this room.

If you are not sure which one, see \cref{sec:sol-Shadowy-Closet}.

\hyperref[sec:route-9]{\emph{Return to route}}

\newpage
\subsection{Attic}\label{sec:req-Attic}

There is no puzzle to solve here, just get the items.

There is a note about the walkie-talkie in \cref{sec:nb-walkie-talkie}.

\hyperref[sec:route-10]{\emph{Return to route}}

\newpage
\subsection{Disheveled Studio}\label{sec:req-Disheveled-Studio}

You can find a library book with relative ease here.

But don't rush off, because there's another important item here that the book will help you to obtain.

For the solution to this room, see \cref{sec:sol-Disheveled-Studio}.

\hyperref[sec:route-10]{\emph{Return to route}}

\newpage
\subsection{Abandoned Nursery}\label{sec:req-Abandoned-Nursery-0}

Examine the vacuum cleaner. The missing component is not in this room.

\hyperref[sec:route-10]{\emph{Return to route}}

\newpage
\subsection{Invasive Library}\label{sec:req-Invasive-Library}

One of these books belongs to the public library.

For the solution to this room, see \cref{sec:sol-Invasive-Library}.

\hyperref[sec:route-10]{\emph{Return to route}}

\newpage
\subsection{Science Tower}\label{sec:req-Science-Tower}

You can pick up two items here that you don't use within the room itself.

You only need one of them,
but I don't want to spoil things by telling you which one, so take them both.

For the solution to this room, see \cref{sec:sol-Science-Tower}.

\hyperref[sec:route-10]{\emph{Return to route}}

\newpage
\subsection{Branching Corridor}\label{sec:req-Branching-Corridor}

Unblock the southeast exit.

For the solution to this room, see \cref{sec:sol-Branching-Corridor}.

\hyperref[sec:route-10]{\emph{Return to route}}

\newpage
\subsection{Shambolic Shack}\label{sec:req-Shambolic-Shack}

There is an item hidden here you need to retrieve.

You should absolutely, definitely save before starting work on this puzzle as,
due to a bug in the \textsc{take back} command (at least up to Release 10)
you can make the game unwinnable. If you get stuck, \textsc{restore} instead.

For the solution to this room, see \cref{sec:sol-Shambolic-Shack}.

\hyperref[sec:route-11]{\emph{Return to route}}

\newpage
\subsection{Rec Room}\label{sec:req-Rec-Room}

You need a key from another room to make progress here.

At the end you will find an item that clearly belongs in a different room.

For the solution to this room, see \cref{sec:sol-Rec-Room}.

\hyperref[sec:route-11]{\emph{Return to route}}

\newpage
\subsection{The Meatpacking Plant}\label{sec:req-The-Meatpacking-Plant}

You need something designed to cleave meat to complete your task here.

Once you have, you'll find an item containing a clue you'll need for another room.

For the solution to this room, see \cref{sec:sol-The-Meatpacking-Plant}.

\hyperref[sec:route-11]{\emph{Return to route}}

\newpage
\subsection{Curiosity Shop}\label{sec:req-Curiosity-Shop}

Contrary to how things normally go in such shops, you need to bring something
curious and come away with something perhaps a little more mundane.

For the solution to this room, see \cref{sec:sol-Curiosity-Shop}.

\hyperref[sec:route-12]{\emph{Return to route}}

\newpage
\subsection{Amorphous Tunnel}\label{sec:req-Amorphous-Tunnel}

You need a key from another location to open the west door. Keep hold of it.

If you are not sure which one, see \cref{sec:sol-Amorphous-Tunnel}.

\hyperref[sec:route-12]{\emph{Return to route}}

\newpage
\subsection{Narrow Straits}\label{sec:req-Narrow-Straits}

You need a key from another location to open the west door.

You need a second key from elsewhere to retrieve an item hidden in this area.

For the solution to this room, see \cref{sec:sol-Narrow-Straits}.

\hyperref[sec:route-12]{\emph{Return to route}}

\newpage
\subsection{Subterranean Tunnel}\label{sec:req-Subterranean-Tunnel}

If you are following the recommended route,
you just solved the required aspect of this room from the other side.
Otherwise the comments about the
\hyperref[sec:req-Amorphous-Tunnel]{Amorphous Tunnel} apply.

There is also the matter of the woman.
You should probably find out what's going on with her,
otherwise you'll forever be wondering.

\hyperref[sec:route-12]{\emph{Return to route}}

\newpage
\subsection{Backwater Public Library}\label{sec:req-Backwater-Public-Library}

I hope you already worked out on your first visit what you need to do here:
return Peter's overdue library books and take out a new one.

Sounds quite mundane when I put it like that, doesn't it?

For the solution to this room, see \cref{sec:sol-Backwater-Public-Library}.

\hyperref[sec:route-12]{\emph{Return to route}}

\newpage
\subsection{The Invisible Worm}\label{sec:req-The-Invisible-Worm}

There's an item here you need to purloin.

You could, possibly, solve this room as a standalone puzzle,
but if you are in possession of insider knowledge it will save an awful lot of guesswork.

For the solution to this room, see \cref{sec:sol-The-Invisible-Worm}.

\hyperref[sec:route-12]{\emph{Return to route}}

\newpage
\subsection{Greenhouse}\label{sec:req-Greenhouse}

The main item you need to solve the puzzle is already in the room,
but you need an item from elsewhere to fix it.

The prize for completing this room is a cardboard box, or rather,
what's inside it.

For the solution to this room, see \cref{sec:sol-Greenhouse}.

\hyperref[sec:route-13]{\emph{Return to route}}

\newpage
\subsection{Court}\label{sec:req-Court}

By now you should have all the information you need to solve the puzzle.
Sorry for making you wait so long.

Having solved it, you will release an important artefact.

For the solution to this room, see \cref{sec:sol-Court}.

\hyperref[sec:route-13]{\emph{Return to route}}

\newpage
\subsection{Abandoned Nursery}\label{sec:req-Abandoned-Nursery}

The main item you need to solve the puzzle is already in the room,
but you need an item from elsewhere to fix it.

There is an item hidden here that you need for\xelip
well, take a guess. It's not the Observatory, it's not the Shack Exterior,
it's not the Train Station Lobby, it's not the Steeple\xelip

For the solution to this room, see \cref{sec:sol-Abandoned-Nursery}.

\hyperref[sec:route-13]{\emph{Return to route}}

\newpage
\subsection{Sitting Room}\label{sec:req-Sitting-Room}

You need the letter opener to solve this room.

You should leave with something else metallic instead.

For the solution to this room, see \cref{sec:sol-Sitting-Room}.

\hyperref[sec:route-13]{\emph{Return to route}}

\newpage
\subsection{Steeple}\label{sec:req-Steeple}

While you could brute force this room,
you do have the \hyperref[sec:nb-Pantry]{clue from the Pantry} to help you.

You need to make a note of the textual description of the correct pattern.

For the solution to this room, see \cref{sec:sol-Steeple}.

\hyperref[sec:route-13]{\emph{Return to route}}

\newpage
\subsection{Train Station Lobby}\label{sec:req-Train-Station-Lobby}

You need a key from another location to open the east door.

If you are not sure which one, see \cref{sec:sol-Train-Station-Lobby}.

\hyperref[sec:route-13]{\emph{Return to route}}

\newpage
\subsection{Station Security Room}\label{sec:req-Station-Security-Room}

You need to obtain a key from this location.

For the solution to this room, see \cref{sec:sol-Station-Security-Room}.

\hyperref[sec:route-13]{\emph{Return to route}}

\newpage
\subsection{Shack Exterior}\label{sec:req-Shack-Exterior}

You just need to unlock the door with tarnished brass key
from the Station Security Room.

\hyperref[sec:route-13]{\emph{Return to route}}

\newpage
\subsection{Inside the Shack}\label{sec:req-Inside-the-Shack}

Look, you've found Peter! Hurrah.

Perhaps you don't need that ritual from the grimoire after all.

If you have only been taking the thing you need from each room,
you should have only one item left that is (\emph{a}) unused and
(\emph{b}) not mentioned in the grimoire. Use that.

For the solution to this room, see \cref{sec:sol-Inside-the-Shack}.

\hyperref[sec:route-13]{\emph{Return to route}}

\newpage
\subsection{Observatory}\label{sec:req-Observatory}

Ah well, it turns out you really do need the ritual.
To recap, that means you need

\begin{itemize}
\item the horn of the black goat (from the Court);
\item the grimoire (from Backwater Public Library);
\item knowledge of the right constellation (from the Steeple);
\item a divine pustule (from the Narrow Straits);
\item Peter's most treasured memento.
\end{itemize}

One of the aspects of the puzzle is to work out what the memento is.

For the solution to that little mystery,
and for a bit of help assembling the right constellation,
see \cref{sec:sol-Observatory}.

The page after that section deals with the final conclusion of the game.

\newpage
\section{Room solutions}\label{sec:sol}

\subsection{Train Station Restroom}\label{sec:sol-Train-Station-Restroom}

Open the combination lock using the code in the closet
to get a jar containing a shape-shifting creepy crawlie.

\hyperref[sec:route]{\emph{Return to route}}

\newpage
\subsection{Exterior of Train Station}\label{sec:sol-Exterior-of-Train-Station}

Open the bin and retrieve the doll and the book.

Read the book to learn coffee divination. You can drop the book after that.

For more information about the doll, see \cref{sec:nb-doll}.

Turn the doll's head to the scowling face, and pull the string until the woman becomes alarmed.
Give the insect jar to the woman to scare her away.
Unfortunately, she doesn't take it with her; feel free to drop it.

\hyperref[sec:route]{\emph{Return to route}}

\newpage
\subsection{Milkweed}\label{sec:sol-Milkweed}
Push the shack over to reveal an altar. Examine the altar, then the cavity.

Get the milkweed leaf that works as a mask. Wear it as protection.

Lie on the altar. Examine the shelf and read the literature provided.
If you want to know which pieces of information to make a note of,
see the note in \cref{sec:nb-Variegated-Court}.

To solve the optional puzzle, get the sphere, examine it and the slot.
You need a coin.

Search the tracks to find one. Put it in the sphere to obtain an athame (dagger).
Use it to cut the tendril attaching the earworm to your brain.

\hyperref[sec:route]{\emph{Return to route}}

\newpage
\subsection{The Churchyard}\label{sec:sol-Churchyard}

Get the flashlight and retain it for later.

Try to open the mausoleum door. You need more leverage.

Examine the fence. You'll find something you can use like a crowbar (\textsc{open door with}).

\hyperref[sec:route]{\emph{Return to route}}

\newpage
\subsection{Mausoleum}\label{sec:sol-Mausoleum}

There's a book. It seems you should read it, and you can't read it without taking it.
(And you do need to take it away with you.)

The mirrors are worth a few looks each but play no part in the solution.

This is one of those occasions where the game tries to put you off doing what you need to do.
Just keep on taking the book until it gives up trying to fight you.

If this is your first library book, you acquire a haunting presence at this point.
If you want to know what that's all about and if you should be worried,
see \cref{sec:nb-library-books}.

\hyperref[sec:route]{\emph{Return to route}}

\newpage
\subsection{The Dim Recesses of the Forest}\label{sec:sol-The-Dim-Recesses-of-the-Forest}

When the leaves arrive, search them for a stone.

Skim the stone.

Uh oh. Something scary will happen.

If you hold your nerve, you get a locket out of it.

Examine it. Piece together your experience for a key fact about another of those
people I told you about in \cref{sec:nb-Variegated-Court}.

\hyperref[sec:route-1]{\emph{Return to route}}

\newpage
\subsection{Shack Exterior}\label{sec:sol-Shack-Exterior-0}

Reassemble the scraps into a page.

Give it to the doll in exchange for the journal.

Read it.

\hyperref[sec:route-1]{\emph{Return to route}}

\newpage
\subsection{Backwater Public Library}\label{sec:sol-Public-Library-0}

Examine the jacket to get a library card.

Give the card to librarian to get a list of all books you need to return.

It's probably safest to leave the card here; you won't want to lose it on your
travels.

\hyperref[sec:route-1]{\emph{Return to route}}

\newpage
\subsection{Estate Agent's Office}\label{sec:sol-Estate-Agent's-Office}

For interest, examine the telegram and the backpack.

Ask the woman about Cragne Manor to reveal the \emph{Twin Hearts} book you need to reclaim.

Now you've established a rapport, you can use the shortcut \enquote{\textsc{a}
\emph{subject}} for \enquote{\textsc{ask woman about} \emph{subject}}.

You can't steal the book while she's looking, so ask about another property,
then try to steal it while she's at the filing cabinet.

You now know what you're up against. On this occasion it is not madness to do
the same thing over and over again, expecting different results.
In total you need to ask about 6 properties to hide the book again,
and a seventh in order to steal the book.

Here are some of places you can ask about:

\begin{multicols}{2}
\begin{itemize}
\item The three properties listed in the pamphlet
\item The station
\item The church
\item The library
\item The estate agent
\item The pub
\item The jail
\item The meatpacking plant
\item The curiosity shop
\end{itemize}
\end{multicols}

I suppose the intention here is for you to chip away at this room as you
progress through the game, but it's more convenient to do it in one go.

If you just concentrate on getting the book, though, you'll miss out on a wealth of
entertaining interaction. Examine the folders, and ask questions about the content.
The information you can learn about the pub is surprisingly relevant,
for example. But do not ask about the Gulf Stream unless you really enjoy scrolling.

Once you're outside, read the book to the end, then see \cref{sec:nb-cyphered-message}.

\hyperref[sec:route-1]{\emph{Return to route}}

\newpage
\subsection{River Walk}\label{sec:sol-River-Walk}

Get the pole and fish the river with it to reveal a lobster trap.

You need to cut the twine with something in this area.
The best hint for this comes if you try to cut the twine with something from a different area.
You could, for example, go to Milkweed, put the imaginary athame in your backpack,
bring it back here and try to use it on the thin loops.

There is a more obscure hint in the soggy tome.

The solution is to put the sick lobster out of its misery with the rock.
On doing so, you will be rewarded with a sharp shard of shattered carapace.
Use it to cut the thin loops of twine,
after which you can open the trap and get the ID card.

You can leave the rock, the shard and the pole behind.

\hyperref[sec:route-1]{\emph{Return to route}}

\newpage
\subsection{Under the Bridge}\label{sec:sol-Under-the-Bridge-0}

Examine the river.

Examine the payphone.
If you try it enough times, you should be able to find a phone number among the graffiti,
flanked by loosely drawn stars.
Search the coin return to find a quarter.

When you make your call, examine the river again to discover a clue about another Alderman.

\hyperref[sec:route-1]{\emph{Return to route}}

\newpage
\subsection{Railway Platform}\label{sec:sol-Railway-Platform}

Here is one solution. You may find others.

Examine the rat to discover a brass winding key.

Use the ID card to open the Lost and Found locker.

Examine Peter's suitcase to see that it has a combination lock.
Examine Peter's wristwatch for a clue.

Examine the mystery box, then compare what you find to other features of the platform:
you are looking for a time of day.

Stand on the bench and use the brass winding key to set the station clock to that time.

Once you have opened the box, get the coin and compare it to the coin slot on the vending machine.

Use the coin to retrieve the bubble, and therefore the eyepiece.

The eyepiece is all you need to take from this room,
but what sort of wife would you be if you left Peter's things lying around the place?

\hyperref[sec:route-2]{\emph{Return to route}}

\newpage
\subsection{The Old Well}\label{sec:sol-The-Old-Well}

Examine the oak, and look in the knothole to find a parcel containing a mouldy journal and a battery.
The battery fits the eyepiece.
The journal explains both the mechanics of the puzzle and what the eventual solution is,
albeit cryptically.

Examine the puddle, the wall, the root and the quartz.
Then wear the eyepiece and examine them again.
Create microphids by prodding the cylinder,
then three coloured arches in succession, then the wooden arch.
Happily, the game gives you shortcuts to make this process less tedious.

There are 27 types of microphid you can summon (C~=~cliffs, F~=~forest, O~=~ocean, P~=~plain):

\begin{itemize}
\item
These ones will kill you. Use \textsc{undo} then prod them to remove them:\newline
\textbf{CFC},\enskip
\textbf{CFO},\enskip
\textbf{COC},\enskip
\textbf{CPO},\enskip
\textbf{FCO},\enskip
\textbf{FPO},\enskip
\textbf{OCO},\enskip
\textbf{OCP},\enskip
\textbf{OFO},\enskip
\textbf{OPO}.

\item
These ones will wander off, each in their own special way:\newline
\textbf{CFP},\enskip
\textbf{COP},\enskip
\textbf{CPC},\enskip
\textbf{FCP},\enskip
\textbf{FOC},\enskip
\textbf{FOF},\enskip
\textbf{FOP},\enskip
\textbf{FPC},\enskip
\textbf{OFC},\enskip
\textbf{OFP},\enskip
\textbf{OPC}.

\item
The first three of these you summon will stay peacefully with you:\newline
\textbf{COF},\enskip
\textbf{CPF},\enskip
\textbf{FCF},\enskip
\textbf{FPF},\enskip
\textbf{OCF},\enskip
\textbf{OPF}.
\end{itemize}

As per the hint in the journal, summon the slime ape (FPC) to coat the well with glue, and three friendly microphids to protect you.

When you have done all the summoning you wish to do,
summon the fearsome pteranodon (FCO).
Because of your precautions, it will fly off with the well cap instead of killing you.

\hyperref[sec:route-2]{\emph{Return to route}}

\newpage
\subsection{Bridge}\label{sec:sol-Bridge}

If you want to play this realistically, you might want to seal everything
(except the flashlight) safely inside your backpack for the duration of the puzzle.
But I don't think it matters.

Examine the metal thingy. Open it and throw the rope into the river.

To get the full experience, try climbing down into the river unaided,
then solve the first issue, then the second issue.
If however you want to complete the room in the fewest possible moves,
explore the bridge thoroughly before descending
and you'll find you've already solved the second issue.

To solve the first issue, as you hang under the bridge,
note the lung-like appearance of the weed and the strange behaviour of the bat.
Get the weed, eat its spores, then go down into the river.

To solve the second issue, examine the sculpture on the bridge.
Replace the dead flashlight battery with a combination of
(flakes of) the dry vegetation in the brazier and the liquid in the pool.

Once you are safely in the water, examine the bones.
Enter the skeleton and hit the flat bones to release them.

The bones are too heavy to carry up the rope,
so put the bones in the claw trap and close it.
Climb up then pull the rope up after.

Use the bones to repair the bridge.

\hyperref[sec:route-2]{\emph{Return to route}}

\newpage
\subsection{Outside Pub}\label{sec:sol-Outside-Pub}

Pick up the rusty leg and put it in the handle.

You can then pull the handle enough to break open the box.

\hyperref[sec:route-2]{\emph{Return to route}}

\newpage
\subsection{Backwater Jail}\label{sec:sol-Backwater-Jail}

Examine the computer and monitor to learn the location of the evidence key.

Read the paperwork on the clipboard to get the right name.

Open the file cabinet then \textsc{get} and \textsc{read} the name.
(The evidence for this case is a library book. I wonder if it's on your list?)

Use the aluminum key to open the locker\xelip

\hyperref[sec:route-2]{\emph{Return to route}}

When you are next able, look in the locker and retrieve the book. Read it.

\hyperref[sec:route-2]{\emph{Return to route}}

\newpage
\subsection{Padded Cell}\label{sec:sol-Padded-Cell}

Examine the chest and the lightbulb, and check your inventory.
Your stuff is in the chest and you're in a strait jacket.

Push the chest to move it under the light.

Stand on the chest and \textsc{hit} the wire mesh (to get it with your teeth
-- \textsc{get} doesn't work but several other verbs do, e.g. \textsc{headbutt}).

Stand back on the chest and \textsc{hit} the lightbulb.

Get down, put the shard in the crack in the chest and cut yourself free.

You can then retrieve your stuff from the chest.

Examine the walls to reveal a brass button. Examine it more closely.
Turn it to reveal a trophy and the door lock switch.

You don't need the trophy itself, but you do need to examine it and make a note
of the description.

Pressing the switch doesn't work; you have to turn it off.

You can now go south and finish off whatever you were doing in
\hyperref[sec:req-Backwater-Jail]{Backwater Jail}.

\hyperref[sec:route-2]{\emph{Return to route}}

\newpage
\subsection{Outside the Plant}\label{sec:sol-Outside-the-Plant}

If you try entering the plant, you'll find an unseen presence stops you.

Push the pig corpses to reveal a sheep. Examine it.

\textsc{Open} or \textsc{cut} it to spill the guts.
Examine the guts, then examine the flies.
You can't read the guts yourself, but the flies are loud, so listen to them.

Examine the walls (or patterns, or block) and construct the magic name.
If you are not sure how, see the clue at the bottom of the page.

Use this as a command to release the beast from the hole into the sky.

\hyperref[sec:route-3]{\emph{Return to route}}

\vfill
What letters are missing from the name of the building?

\newpage
\subsection{Dusty Office}\label{sec:sol-Dusty-Office}

Search the detritus repeatedly. You will find some interesting information and
eventually a key.

Unlock the cabinet with the key to get a notebook.

You should be able to work out the familiar of another Alderman.

\hyperref[sec:route-3]{\emph{Return to route}}

\newpage
\subsection{Bathroom of the Meatpacking Plant}\label{sec:sol-Bathroom-of-the-Meatpacking-Plant}

Examine everything outside the stall thoroughly.

Enter the stall and note the missing flush handle.
Come back out and try to operate the shower or sink.
You will get what you need, but release the boneless horse.
Lock yourself in the stall to survive.

Read Juggs and the note that falls from it to get the hint about the candle
(and some things to look for). Repair the toilet.
Put the hand candle in the pentagram and light it,
then sit on the toilet and flush it.

In the space toilet, examine the corpses to get the janitor's key.
Get the blue journal.
Get and examine the scroll for instructions and a list of items to find.
Examine and crank the phonograph to realise you have disposable wax.
Examine the baby names to realise you have a disposable book.
Examine the fireplace, light it with the baby names for kindling.
Put the cylinder on the hearth to melt it, then sculpt it into a horse poppet.
Flush.

Back in the first stall, use the key to open the closet,
and get the fur and the foot candle.
Swap over the foot candle and the hand candle, and light the foot one.
Flush.

In the Cambrian toilet, get and read Ed's journal,
and get the key to the blue journal.
Unlock and read that too;
eventually you will discover the information you need about an Alderman.
Take the can of Mountain Dew and milk the trilobite into it.
Flush.

Put everything except what you need for the spell in your backpack and close it.
Complete the potion by pouring pepto-bismol into the Mountain Dew can. Drink it.
I recommend saving at this point.

You can now perform the ritual:
put the hand or foot candle then the horse poppet on the pentagram,
wave the fur,
put Juggs on the pentagram and attack it with the athame,
then use the magic word.

It doesn't quite work so flush the horse away.

As you leave, you get a different cutscene depending on which candle you used.

\hyperref[sec:route-3]{\emph{Return to route}}

\newpage
\subsection{Cragne Family Plot}\label{sec:sol-Cragne-Family-Plot}

Open the eastern columbarium to find a silver urn and a key.

Use the urn to bail out the water in the grave (e.g.~\textsc{fill urn},
\textsc{empty grave}, \textsc{bail grave}).

Descend and dig west to find a pewter box.

Go back up and unlock the box with the urn key to get the book.

\hyperref[sec:route-3]{\emph{Return to route}}

\newpage
\subsection{Back Garden}\label{sec:sol-Back-Garden}

Examine the building: the way in is secured with a padlock.

Examine the shelf by the building to find something you can use as weedkiller.

Use this to kill the poison ivy and release the shears.

Examine the bird bath. Read the inscription.

Examine the pond and clear the vines from it.
Examine the contents thoroughly to discover a screwdriver.

Examine the fountain, then the cherub, then look in the fountain.
Use the screwdriver to free the key.

Unlock the padlock with it.

\hyperref[sec:route-3]{\emph{Return to route}}

\newpage
\subsection{Shambolic Shack}\label{sec:sol-Shambolic-Shack-0}

Examine the shelving. Get the gloves to find a hefty, rust-streaked iron key.

Keep the key but put the gloves back on the shelf.
They are too manky to carry around with you.

\hyperref[sec:route-3]{\emph{Return to route}}

\newpage
\subsection{Under the Bridge}\label{sec:sol-Under-the-Bridge}

Use the iron key from the Shambolic Shack to open the hatch.

Before going down, put everything you're carrying (except the flashlight)
in the backpack and close it.

\hyperref[sec:route-4]{\emph{Return to route}}

\newpage
\subsection{Tunnel Entrance}\label{sec:sol-Tunnel-Entrance}

Examine the masonry and the walls. Examine the opening. Get the flask.

Examine the rags.

Open the flask and give it to the hobo so he drops something,
then examine the remaining features of the tunnel
while you wait for him to fall asleep again.

Look for what the hobo dropped and grab it. Use it to open the hatch.

Opening the hatch reveals a fusebox, two candles and a mallet.

Examine the fusebox. Switch on the switch.
(It is probably a bug that the switch is described as off even when it is on.)

If you weren't carrying the flashlight,
you can now go back to the Subterranean Tunnel and retrieve the things you dropped.

Once you're back,
compare the fusebox with the sockets in the opening, and note the position of the fuses.
That will tell you what to do with the candles.

Examine the picture (possibly for the second time): this time you will see a clue.
Use the mallet to hit some of the bricks. It helps to know about musical scales,
but you'll be able to work it out regardless.

\hyperref[sec:route-4]{\emph{Return to route}}

\newpage
\subsection{Small Chamber}\label{sec:sol-Small-Chamber}

Examine all the crates. Examine the padlock.

Look in the large crate, repeatedly, to find a phonograph, a loudspeaker horn
and a record. Assemble the phonograph, wind it up and play the record.

Open the medium crate. Examine the newspaper, then read it until you are able to
calculate the date of Ceecee's death. (You may find a dedicated \href{https://www.timeanddate.com/date/dateadd.html}{date calculator} helpful for this.)
You then need to translate this into a four digit number (M-D-YY).

Make sure you do not have the suitcase or any other combination lock visible
(zip it away in the backpack if you have it with you)
and feed the number into the padlock one digit at a time:
\textsc{set knob to}, \textsc{turn knob to}, \textsc{dial knob to} or simply
\textsc{dial} should all work.

Open the small crate to reveal a paperback library book.

\hyperref[sec:route-4]{\emph{Return to route}}

\newpage
\subsection{Tiny Windowless Office}\label{sec:sol-Tiny-Windowless-Office}

Examine the papers to get the large brass key.

Read the note and open the parcel.

Examine the box, the panel and the buttons.

Press the buttons corresponding to the lid illustrations in the right order
-- if you get stuck, see the hint at the bottom of the page --
then the large button, to get Luther's tiny leather journal.

Examine the journal and the embossed design, then read the contents.
That's another Alderman for your list.

\hyperref[sec:route-4]{\emph{Return to route}}

\vfill

The buttons represent the numbers 0--4.
The animals all have different numbers of something.

\newpage
\subsection{Mudroom}\label{sec:sol-Mudroom}

Get the teapot. Read the inscription. The date refers to a real historical incident involving
that ship, which nicely ties in with what you will learn in the room where you use it.

Unlock the front door with the large brass key.

Everything else is decoration: have fun playing with it.

\hyperref[sec:route-5]{\emph{Return to route}}

% \newpage
% \subsection{Foyer}\label{sec:sol-Foyer}
\newpage
\subsection{Landing at the Bottom of Stairs}\label{sec:sol-Landing-at-the-Bottom-of-Stairs}

Stand on the bookshelf to look out the window then get down.

Look in the newly uncovered alcove to get another library book.

\hyperref[sec:route-5]{\emph{Return to route}}

\newpage
\subsection{Top of Stairs}\label{sec:sol-Top-of-Stairs}

There is a pink-bound library book in the cupboard under the sink.

When you read the book, stay high, and find and eat the hunter.

\hyperref[sec:route-5]{\emph{Return to route}}

\newpage
\subsection{Nursery -- Hillside Path -- Carol's Room}\label{sec:sol-Nursery--Hillside-Path--Carol's-Room}

These rooms can be a bit judgemental about how you solve them, so if you care
about such things you might want to save so you can replay them with the advantage
of hindsight.

Ask Carol about as much as possible before being evicted.

Return to Hillside Path.
Talk some more to Christabell.
Receive her mark and learn a magic spell for destroying the rune book
and another for transportation.

Return to Carol's room.
Drat, the rune book isn't here. You have to accept Carol's mark before you can leave again.

Return to Hillside Path.
Talk some more to Christabell to learn your next task.

Return to Carol's room.
When you've nosed around enough, but keeping some questions in reserve,
destroy the binder with your spell.

If there is one flaw with this scene, it's that you never know if Carol is
distracted or not. Apparently (and counter-intuitively) she is distracted
when answering one of your questions. Therefore, ask Carol about something
then immediately open the window (twice).

If you want to avoid some heartbreak, jump out the window. If you want more drama,
hang around.

After the showdown, you will see your own rune book.
Interact with it to receive a worn out, decaying photograph with a riddle on it.
There is a copy of the riddle in \cref{sec:nb-picture-riddle}.

\hyperref[sec:route-5]{\emph{Return to route}}

\newpage
\subsection{Hallway South}\label{sec:sol-Hallway-South}

Peel the wallpaper and go through the hole to read some source code for the room.

This will tell you where to find the typescript.

\hyperref[sec:route-5]{\emph{Return to route}}

\newpage
\subsection{Library}\label{sec:sol-Library}

Examine and take the picture to reveal a safe.

Look through the paperbacks. One book is noticeably the odd one out.

Read it to find a key to open the desk.

This will tell you how to find the safe combination using the other obvious feature in the room.

Inside the safe is another (public) library book.

\hyperref[sec:route-5]{\emph{Return to route}}

\newpage
\subsection{Gallery}\label{sec:sol-Gallery}

There are a few ways of clearing the mirror away from the north door.

These are the ones I found:

\begin{itemize}
\item Just keep pushing or pulling the mirror.
\item Smash the mirror.
\item Follow the hint from the fortune cookie and slash open the picture of the child (if you examined the mirror enough, the game will automatically give you a shard of glass to do this with).
\end{itemize}

You may be able to find a better solution.

\hyperref[sec:route-6]{\emph{Return to route}}

\newpage
\subsection{Music Room}\label{sec:sol-Music-Room}

For a map of the locations you can visit, see \cref{sec:nb-Music-Room}.

In the Girl's Bedroom (was this also Carol's room?),
observe the ritual until you are evicted.

In the Living Room, examine and get the phone to listen to the message.

In the Playground, examine the seesaw and all its components.
You should find chalk and a note. Read the note, and examine the thing it mentions.
Listen to the message you are given.
It is important, so you can revisit the Playground to hear it again, or read the note.

In the Living Room, it is now night, and there is something you can \enquote{pick up}.
You have everything you need to \enquote{get ears}
(remember you are reliving Francine's experiences).
Wear them.

At the Club Backstage, examine the literature on display.
Notice the names of the bands on the gig poster.
Examine the name that is most familiar, apart from Francine.

In the Girl's Bedroom, give the chalk to Francine (you die otherwise).
Leave the closet, open the window to air the room, and receive the plane\slash picture.
Examine it thoroughly. Retrieve the chalk again.

In L'oreille, examine and look in the kaleidoscope.
Examine the wall and keep listening to discover Francine's colour/rank.

In the Basement, read the literature on offer.
Examine the freezer, open it, get the object inside (twice).
You should be able to find another ring.

In the Forestland, draw a chalk circle and put both rings on the corpse's fingers.

Back in the Music Room, you'll be given another paper plane.
Read it to learn Francine's familiar.

You can safely drop anything you picked up from this area.

\hyperref[sec:route-6]{\emph{Return to route}}

\newpage
\subsection{Cragne Family Plot}\label{sec:sol-Cragne-Family-Plot-1}

The white key from the Court opens the crypt.

\hyperref[sec:route-6]{\emph{Return to route}}

\newpage
\subsection{Crypt}\label{sec:sol-Crypt}

Read the coffin to discover another Alderman colour/rank.

Open it to discover his familiar and a long wooden key.

\hyperref[sec:route-6]{\emph{Return to route}}

\newpage
\subsection{Town Square}\label{sec:sol-Town-Square}

Assuming the clue you read on the worn out, decaying photograph matches what
is in \cref{sec:nb-picture-riddle}, rotate the rings of the emblem like so:

\begin{adjustbox}{center}
\begin{tikzpicture}[baseline=0pt]
\draw[Gray] (0,0) circle[radius=1];
\draw[Gray] (0,0) circle[radius=2];
\draw[Gray] (0,0) circle[radius=3];
\draw[Gray] (0,0) circle[radius=4];

\foreach \x/\xtext in
{90/n, 0/e, 270/s, 180/w}
\node[align=center,font=\sffamily\scshape] at (\x:4.2) {\xtext};

\foreach \y/\xtext [evaluate=\y as \x using \y-120] in
{90/{Black-\\bird}, 30/Tree, 330/Sparrow, 270/Cube, 210/Woman, 150/Feather}
\node[align=center] at (\x:3.5) {\xtext};

\foreach \y/\xtext [evaluate=\y as \x using \y-60] in
{90/Daan, 30/Shi, 330/Aak'ee, 270/Hai, 210/Tlèè, 150/Ji}
\node[align=center] at (\x:2.5) {\xtext};

\foreach \y/\xtext [evaluate=\y as \x using \y+120] in
{90/Pig, 30/Eye, 330/{Wood-\\pecker}, 270/Cross, 210/Fish, 150/Eagle}
\node[align=center] at (\x:1.5) {\xtext};

\node[rotate=-60] at (30:1) {Δ};

\node[align=center] at (0:0) {Bird\\claws};
\end{tikzpicture}
\end{adjustbox}

Say \textsc{ahe'hee}.

Get the amulet from the hole and examine it.
The clue is more important than the object.

\hyperref[sec:route-7]{\emph{Return to route}}

\newpage
\subsection{Drinking Fountain}\label{sec:sol-Drinking-Fountain}

There should be a ghost here by now, the one that has been haunting you since
you first encountered a library book.

Pull the ivy to reveal a second fountain. Read the sign. Shudder.

Demolish the smaller fountain to receive the reading glasses.
There's no mystery to these: if you wear them,
they will tell you if there is an unread library book where you are.

Examine the hole you have revealed to find another library book.

\hyperref[sec:route-7]{\emph{Return to route}}

\newpage
\subsection{Church Exterior}\label{sec:sol-Church-Exterior}

Unlock the door with the long wooden key from the Crypt.

\hyperref[sec:route-7]{\emph{Return to route}}

\newpage
\subsection{Church Lobby-Space (Narthex)}\label{sec:sol-Church-Lobby-Space}

This area gives you strong hints, and in places runs on rails. Go with the flow.

If you examine the photos and the students in them, you'll see a girl with lipstick.
You'll also find an interesting article.
The room name changes to Narthex once you discover the proper name for it.

Go into the restroom. Examine the mirror, and wear the lipstick to experience
part one of the flashback. Examine the mirror again to confirm.

Give the lipstick to the nun to end the flashback. Examine the mirror again.

Wear the red shoes to experience part two. (Examine the mirror again to confirm.)
Look under the ladder to find the hiding place, and hide the book to end the vision.
(Examine the mirror again to confirm.)

Retrieve the library book and read it.
Look around to see that a noose has appeared.
Go up for one more vision.

\hyperref[sec:route-7]{\emph{Return to route}}

\newpage
\subsection{Chapel}\label{sec:sol-Chapel}

You can optionally score points for various actions here, e.g. finding the pencil,
putting the teeth in the bottle.

Examine the plate and the lockbox. Try to take the tablecloth.
Recall the item you picked up in the Circular Room.

When you have worked out how to open the lockbox, get the crooked dagger.

Unlock the south door with the long wooden key.

\hyperref[sec:route-7]{\emph{Return to route}}

\newpage
\subsection{Church Basement}\label{sec:sol-Church-Basement}

Examine the TV and tapes for a basic idea of what you need to do,
but try to turn on the TV and you'll find it's unplugged;
the socket is hidden somewhere.

Examine the mannequin, its eye and neck. Push the neck to cut yourself on it.
You should now be able to get the TV working.

Use the clue from the amulet to find a message (remember to set the counter correctly);
there's a copy of it in \cref{sec:nb-Church-Basement}.

\hyperref[sec:route-7]{\emph{Return to route}}

\newpage
\subsection{Dining Room}\label{sec:sol-Dining-Room}

Examine the calendar while you wait for the cut scene to end.

Turn on the machine and keep turning the dial to view how the site looked in the past.

\hyperref[sec:route-8]{\emph{Return to route}}

\newpage
\subsection{Kitchen}\label{sec:sol-Kitchen}

Open the door in the floor with the small rusty iron key from the Study.

The books here are quite interesting but you don't need them.
Having said that, there is another room where, if you are being cautious,
you might feel more relaxed using these ones than those from the library.

\hyperref[sec:route-8]{\emph{Return to route}}

\newpage
\subsection{Basement}\label{sec:sol-Basement}

Get the jar of keys and the screw jar. Open them.
(You don't strictly speaking need the screw jar,
but it's nice for keeping the slimier items in.)

Look under the carpet and pull it aside to reveal a hatch.

Go down to enter the Cragne Library Forbidden Annex.
(Do not go north from here.)

Examine everything (especially the bat) and try to open the cabinet.
Now, you could give away all your books, or\xelip

Use the sinister iron key to unlock the cage.
The ghost will chase the bat giving you free rein with the books.
Don't be afraid to ring the bell to call Baines back for questioning,
he only stays for two turns.
For completeness, ask him about the author of every book you read.

Get, examine and read three books to reveal a name scrawled in pencil.
Ring the bell and ask the ghost about it (use a straight apostrophe, not a curly one).

Read a few more books.
I recommend \emph{Mysteries of the Red City} followed by \emph{The Searcher in Darkness}
followed by \emph{Mysteries of the Red City} again.
You should stumble across a ritual you can perform, but lacking the magic word.

Keep reading until you stumble across the magic word. (If you miss any of these
details, there are copies in \cref{sec:nb-Cragne-Library-Forbidden-Annex}.)

Ring the bell, and chant the word three times.
Run away from the resulting monster by going up, then to another room.
It is then safe to return.

You will be rewarded with a library book and a vial.
Open the vial to experience a memory regarding Hiram, Silas and Irenius.

\hyperref[sec:route-8]{\emph{Return to route}}

\newpage
\subsection{Cold Storage}\label{sec:sol-Cold-Storage}

While holding the thing you will be projected back into your younger self.

Examine and empty the satchel to get a robe, a bib and some clippings.

Read the clippings.

Touch the eyeball: yes, it's a squid cannon.

Use the robe to stop the squid escaping through the wall.

Use the bib to stop the squid escaping through the floor.

Use the satchel to stop the cat escaping through the catflap.

Put the clippings on the satchel to confuse the cat further.

After that, you will be returned to your body,
and able to read the glyphs for another Alderman.

(Is this the same eye that the eyepiece interfered with?)

\hyperref[sec:route-8]{\emph{Return to route}}

\newpage
\subsection{Pantry}\label{sec:sol-Pantry}

Examine the shelves and follow the hints to get the peaches. (More peaches!)
Open the peaches.

Take the stem of the rotten pumpkin and put it in the peaches.
You will then be able to reconstitute previous meals.
You need never go hungry again.

Put yourself in the peaches to meet a little spark.
Ask it about Edmund to learn a helpful poem about bells
(reproduced in \cref{sec:nb-Pantry}).
Go up to exit.

\hyperref[sec:route-8]{\emph{Return to route}}

\newpage
\subsection{Workroom}\label{sec:sol-Workroom}

Brace yourself, this is another long one.

Examine the mirror to find a delivery note.
Look up the two names for your first magic words.
Use them to find a third name, thence a new word and a mission to work towards.

Use the words you've learned so far and examine the mirror for a chain of clues to follow.
Use words of Invocation and Enlightenment and listen to get another chain of clues.
After following both branches you'll be equipped to squeeze another clue from the mirror.

Use Trance, Invocation and Enlightenment,
and listen for a clue that will set you a more immediate task: to find three glyphs.

Use a season, Enlightenment and Trance to get a clue that will reveal how to get two glyphs.
Use Winter, Trance and melt the glacier to get the third.

This will give you new word. Use it to reveal a way to turn the moon red.
Then use it with Trance to find a fourth dream location.

Freeze the lake to walk to the island and then evaporate it to find the Temple.
There is a clue there that will tell you where to find the Name:
Use Summer, Trance and break open the tomb with the appropriate word.

Return to the Temple and use the name.
This starts a new ritual, so you can use Trance again.

When you reach the gate, check your inventory for a way to open it.

You will find the library book you need, but you can't take it with you.

Hide it where you'll be able to find it again,
then cancel both the inner and outer rituals.

Retrieve the item from its hiding place.

\hyperref[sec:route-8]{\emph{Return to route}}

\newpage
\subsection{Wine Cellar}\label{sec:sol-Wine-Cellar}

Examine the bottles. Drink the odd one out to discover\xelip something.
Smash the bottle to get it.

Empty the cask (twice) to make it light enough to shift out of the way of the door.

\hyperref[sec:route-8]{\emph{Return to route}}

\newpage
\subsection{Laboratory}\label{sec:sol-Laboratory}

Save and have fun losing the game.

Then \textsc{restore}\slash \textsc{take back} and make sure you have the
crooked dagger (from the Chapel) to hand.
I won't tell you what to do with it,
but I'm sure you could take a stab at the answer.

Get the mirror. Examine it and note what it does.

\hyperref[sec:route-8]{\emph{Return to route}}

\newpage
\subsection{Boiler Room}\label{sec:sol-Boiler-Room}

Examine everything. Read the note on the table.
Open the drawer and take what's inside.
Examine and read the journal.

If you want to practice with the equipment, consult the journal about
\textsc{precautions} and \textsc{catch-all}.
Stand on the table and examine the shelf to find more of what you need.
Be sure to take the precautions.

There are lots of places to visit.
The descriptions are great but there's not much to do.

When you have had enough, consult the journal about \textsc{men of power}.
This will start a trail of clues in the journal to the hex code you really need.
If you guess the right code, use it to get the library book.
If you can't get it, there is a more explicit clue, crossword style,
at the bottom of the page.

I did not find a use for the Golden Apple.

\hyperref[sec:route-8]{\emph{Return to route}}

\vfill
Not alive (4)\quad Cow meat (4)

\newpage
\subsection{Courtyard}\label{sec:sol-Courtyard}

Use the handle from the well to crank the knight's head.

Give it an extra twist, then push the crack.
You can now go north.

\hyperref[sec:route-8]{\emph{Return to route}}

\newpage
\subsection{Hallway South}\label{sec:sol-Hallway-South-1}

Unlock the door with the red triangle key from the Wine Cellar.

\hyperref[sec:route-9]{\emph{Return to route}}

\newpage
\subsection{Balcony}\label{sec:sol-Balcony}

Examine the statue to find a sturdy key.

\hyperref[sec:route-9]{\emph{Return to route}}

\newpage
\subsection{Upstairs Hall, North End}\label{sec:sol-Upstairs-Hall-N}

Once it is open, look in the armoire to find a notebook.
Read it repeatedly for clues about an Alderman.

After experiencing a vision concerning the north east doorway,
you will receive a JogMaster that you can listen to.
(Keep wearing it to avoid repeating the vision.)

Unlock the east door with the sturdy key from the Balcony.

\hyperref[sec:route-9]{\emph{Return to route}}

\newpage
\subsection{Master Bedroom}\label{sec:sol-Master-Bedroom}

Examine the bed or lamp. Uh oh.

Look around to shed some light on how to reverse what happened.

\hyperref[sec:route-9]{\emph{Return to route}}

\newpage
\subsection{Shadowy Closet}\label{sec:sol-Shadowy-Closet}

It just takes a little reflection to work out what to do here, but if not the
answer is at the bottom of the page.

\hyperref[sec:route-9]{\emph{Return to route}}

\vfill
Put the mirror from the Laboratory in the frame
to reveal a trap door in the ceiling.

\newpage
\subsection{Disheveled Studio}\label{sec:sol-Disheveled-Studio}

Examine the canvas and table,
then examine the bookshelf to find and read the book for clues on what to do.

Draw (on) the canvas.
Read the book again until you get a papercut, then bleed on the canvas, then kiss it.

Prepare yourself by putting frankincense, cedarwood, and Saviour's Breath (clear oil) on the canvas.

Lastly, put peat venom (teal oil) on the canvas and kiss the canvas.

After the horror show, you will get the slimy key.

\hyperref[sec:route-10]{\emph{Return to route}}

\newpage
\subsection{Invasive Library}\label{sec:sol-Invasive-Library}

Listen to find the odd book out.

\hyperref[sec:route-10]{\emph{Return to route}}

\newpage
\subsection{Science Tower}\label{sec:sol-Science-Tower}

Open the cabinet, and get the peanut bag and the pen. Examine both.

Put the ashtray stand on the table,
then get on the table and lift it by pressing the respective button with the pointer.

Get the egg, and pull the latch to release the skylight.
When the table goes back down, get off and get the stand (to return it to the floor).

You can now put the hamster on the table, open the skylight and lift the table.
Take note of the interval between the thunder and lightning so you can activate the antenna at the right time.

Put the hamster back in the cage. When it completes its work, give it the peanut from the bag.

At some point the egg will hatch. Keep what's inside.

\hyperref[sec:route-10]{\emph{Return to route}}

\newpage
\subsection{Branching Corridor}\label{sec:sol-Branching-Corridor}

Push (or pull) the cacti in the right order to clear a path.

If it gets too hot, move to a different room then come back in.

\hyperref[sec:route-10]{\emph{Return to route}}

\newpage
\subsection{Shambolic Shack}\label{sec:sol-Shambolic-Shack}

As mentioned in \cref{sec:req-Shambolic-Shack}, save before doing anything else.

Search the soil. The game won't let you do it with bare hands, so put on the gloves.

Time is now of the essence. Search the soil and locate the fungicide.
Having done so, immediately drop the gloves and spray your hands.
\emph{Do not spray the gloves.}

(If you carry the gloves around, they will cover the rest of your stuff with
mildew. Sadly, there isn't fungicide to spare to reverse this, so that's why
you mustn't carry the gloves.)

Search the soil again. The thing that lives there wants a snack. Are you carrying food,
however nasty?

Put the dessicated sausage in the soil. Hmm, almost right.

Put the sausage in the gloves then try again.

Wait for the thing to expire, then search the soil again for a tiny brass key.

\hyperref[sec:route-11]{\emph{Return to route}}

\newpage
\subsection{Rec Room}\label{sec:sol-Rec-Room}

There is one cabinet of interest. Examine it, then unlock it with the tiny brass key.

Watch the blank TV for a while until you see an advert for a board game.

Follow the trail of board games until you get the rusty meat cleaver.

\hyperref[sec:route-11]{\emph{Return to route}}

\newpage
\subsection{The Meatpacking Plant}\label{sec:sol-The-Meatpacking-Plant}

Examine the hooks, then what you find on the hooks.

While carrying the rusty meat cleaver, cut the animal.

This gives you the current secret menu for the Invisible Worm.
(Remember Bethany at the Estate Agent's had an out-of-date version?)

\hyperref[sec:route-11]{\emph{Return to route}}

\newpage
\subsection{Curiosity Shop}\label{sec:sol-Curiosity-Shop}

Find an item you like the look of. Offer to buy it.
You won't be able to; in fact you will be asked to donate something.

Give the lady a red herring, and get the ancient key in return.

\hyperref[sec:route-12]{\emph{Return to route}}

\newpage
\subsection{Amorphous Tunnel}\label{sec:sol-Amorphous-Tunnel}

Unlock the east door with the nasty key from the Steeple. Retain the key.

\hyperref[sec:route-12]{\emph{Return to route}}

\newpage
\subsection{Narrow Straits}\label{sec:sol-Narrow-Straits}

Open the sarcophagus with the ancient key from the Curiosity Shop, and retrieve the cyst.

Unlock the west door with the nasty key from the Steeple.

\hyperref[sec:route-12]{\emph{Return to route}}

\newpage
\subsection{Backwater Public Library}\label{sec:sol-Backwater-Public-Library}

Pick up Peter's library card again.

Return all the library books to the librarian to clear Peter's account.

Point to the grimoire to borrow it.
Read it for a ritual to follow, and keep it with you.

\hyperref[sec:route-12]{\emph{Return to route}}

\newpage
\subsection{The Invisible Worm}\label{sec:sol-The-Invisible-Worm}

Examine the walls, and you will see there's something not tied down.
Obviously you are meant to steal it, but not with the barman watching.

Order things off the secret menu.
Two of them are complicated enough to distract the barman.
(When I played, one was mentioned by Bethany way back in the Estate Agent's,
the other by the old timers,
but there's no harm in trying them all: you must be ravenous by now.)

Escape once you've stolen the item.

\hyperref[sec:route-12]{\emph{Return to route}}

% \vfill
% Fiddleheads Three Ways, Ancestor Sandwich.

\newpage
\subsection{Greenhouse}\label{sec:sol-Greenhouse}

Here is one solution. You may find others.

Examine the statue.

Keep going up. On the way, look out for a machete to pick up.

There are things you can cut with it, but you'll find it's blunt,
so sharpen it with the whetstone.

On reaching the top, you'll find a spirit parrot pushes you back down.
Hmm, the statue is of a parrot\xelip

Appease the spirit by cutting the vines from the statue.
When you get to the top again, cut away branches to reveal a cage.
But if you try to get what's inside, the parrots will get you.

Undo, and return to the statue and clean it off with some tool or other
(the whetstone works).

Repeat the exercise and you'll end up with a cardboard box.

\hyperref[sec:route-13]{\emph{Return to route}}

\newpage
\subsection{Court}\label{sec:sol-Court}

Put the right familiars on the right pedestals to unlock the monolith.

\begin{xltabular}[c]{\linewidth}{llll}
\hline
\textbf{Alderman} &
\textbf{Familiar} &
\textbf{Name} &
\textbf{Room} \\
\hline
Cesious & Silverfish & Phyllis & Milkweed \\
Eburnean & Kraken & Eliakim & Dim Recesses \\
Croceate & Eel &  & Under Bridge\\
Fulvous & Duck &  & Outside Pub \\
Puce & Greyhound & Jonathan & Padded Cell \\
Rufous & Rat & Charles & Dusty Office \\
Mazarine & Pontiac Firebird & Konstantine & Bathroom of MP \\
Niveous & Wolverine & Luther & Church Office \\
Xanthic & Peregrine falcon & Francine & Music Room \\
Icterine & Weasel & Harvawell & Crypt \\
Griseous & Cat & Mavis & Cold Storage \\
Fuscous & White antelope & Theo & Upstairs Hall N \\
\hline
\end{xltabular}

Look in the obelisk. Get the horn.

\hyperref[sec:route-13]{\emph{Return to route}}

\newpage
\subsection{Abandoned Nursery}\label{sec:sol-Abandoned-Nursery}

Examine the dollhouse and its interior, you will see it is clogged up with cobwebs.

The vacuum cleaner is missing a component.
Repair it with the component in the cardboard box from the Greenhouse.

Using the vacuum cleaner, clean the dollhouse.
Search the interior to find a silver letter opener.

\hyperref[sec:route-13]{\emph{Return to route}}

\newpage
\subsection{Sitting Room}\label{sec:sol-Sitting-Room}

Examine the mirror to swap identities. The next bit is more or less on rails:
ring the bell, open your letter, give Eustace his letter and letter opener,
and when he attacks you, get the letter opener back.

When you've finished looking around the room and out the window,
look in the mirror again to swap back and receive the chime.

\hyperref[sec:route-13]{\emph{Return to route}}

\newpage
\subsection{Steeple}\label{sec:sol-Steeple}

Examine the mass.

Read the journal and tome, then get and wear the pendant.

Ring the bells in the order specified in the poem from the Pantry
(see \cref{sec:nb-Pantry}).

Observe the pattern of stars and compare it with the pattern shown in the
celestial tome. It should have changed to match.
Make a note of the new description of the astrological sign.

\hyperref[sec:route-13]{\emph{Return to route}}

\newpage
\subsection{Train Station Lobby}\label{sec:sol-Train-Station-Lobby}

Unlock the east door with the slimy key from the Disheveled Studio.

\hyperref[sec:route-13]{\emph{Return to route}}

\newpage
\subsection{Station Security Room}\label{sec:sol-Station-Security-Room}

Examine the article and calculate the date of the incident.

Examine the monitors, then each one in turn.

Examine the tapes. Examine the right date to play the corresponding tape.

This will tell you where to find what Bran stashed.
Read the accompanying note so you know where to use the tarnished brass key.

\hyperref[sec:route-13]{\emph{Return to route}}

\newpage
\subsection{Inside the Shack}\label{sec:sol-Inside-the-Shack}

Ring the chime to trigger another set piece on rails. Don't panic!

When you reach the Book of All Your Days, you get to rewrite your relationship with Peter.

\begin{itemize}
\item
  Turn the page to choose between your first meeting, wedding, and end of relationship.
\item
  North changes your ages when this happened.
\item
  South changes the location\slash environment.
\item
  East swaps over whether you or Peter are making the moves.
\item
  West alters the Lovecraftian horror level.
\end{itemize}

Pull the bookmark to exit and get a photographic record of your choices.

\hyperref[sec:route-13]{\emph{Return to route}}

\newpage
\subsection{Observatory}\label{sec:sol-Observatory}

Use the device to set the right star sign:

\begin{itemize}
\item \textsc{push toggle} to get a random configuration;
\item \textsc{turn crank} to set the first constellation;
\item \textsc{turn dial} to set the second constellation;
\item \textsc{pull lever} to set the relative position (above, under, opposing, in the house of);
\item \textsc{pull pulley} to rotate the projector (ascending, descending).
\end{itemize}

Follow the instructions from the grimoire.
Use the photograph from Inside the Shack as Peter's most treasured memento.

Go through the portal. The solution to the next area is on the
\hyperref[sec:sol-Gulf-of-Nehilim]{next page}.

\newpage
\subsection{Gulf of Nehilim}\label{sec:sol-Gulf-of-Nehilim}

Go in.

\vfill
\subsection{The Great Purple Unknown}

If you lost the game by saying a particular word from \emph{Anchorhead} earlier,
you will now understand where the text for that ending came from.

Examine the gate and the masks.

Examine the masks one by one and do what they suggest.
This seems to heal Peter. You can hug and kiss him now.

Search the gate to find some dials.
Examine the dials, then the left dial and the right dial individually.
This is slightly cruel: even though both are already set to 0,
you have to set them both to 0 yourself to unlock the gate.

Go through to win!

\newpage
\section{Notes and clues}\label{sec:nb}

\subsection{The Pull-String Doll}\label{sec:nb-doll}

I'm not sure it was intended as such,
but the pull-string doll is effectively an in-game hint device (like the coffee).
If you are stuck for something to do,
you can pull the string and the doll will usually identify something in the
environment you can examine or interact with further.

The thing it identifies is random,
so don't worry about it spoiling the solution for you.

It does not matter what mode you operate the doll in:
it is a matter of personal preference whether you want short or long messages.

\hyperref[sec:route]{\emph{Return to route}}

\newpage
\subsection{The Variegated Court}\label{sec:nb-Variegated-Court}

Milkweed contains the first of many clues you will learn about the
Variegated Court. There is a puzzle late in the game that relies on you
remembering what you have learned.

Fortunately, you don't need to memorize every detail of each member's lives.
You only need to know two things: the adjective that describes each Alderman
of the Variegated Court (these are all obscure colours but they possibly also
signify rank), and what form their familiar takes.

There are quite a few, so if you are trying not to rely on this walkthrough too
much, do take notes. Otherwise, I provide a complete list in the puzzle
solution.

\hyperref[sec:route]{\emph{Return to route}}

\newpage
\subsection{Library Books}\label{sec:nb-library-books}

Throughout your travels you will come across many books. The ones you really
need to care about are the library books. Indeed, if you find a way of doing so,
you might find it convenient to keep them separate from the other ones in your
inventory, or just not carry any other books about.

I know you will anyway, but it is important that you read these books.
It is especially important to that presence that started haunting you when you
first saw a library book: so important that it will freeze the library insignia
on every book you read so you can keep track of which ones they are.

The significance of the presence will be become clear later, but it is nothing
to worry about.

\hyperref[sec:route]{\emph{Return to route}}

\newpage
\subsection{The Backpack}\label{sec:nb-backpack}

You don't have a particular carrying limit in this game,
which is nice because you don't have to juggle your inventory.
On the other hand, it does mean you can end up with a lot of stuff,
which can be a problem for a couple of reasons:

\begin{itemize}
\item
  In cases of ambiguity, the things you carry take precedence over the things
  in the room. You may find commands that ought to work are failing because
  the game is doing them to the wrong thing, sometimes without making this
  clear to you.
\item
  You may find it hard to remember, say, which books are library books and
  which contain clues, or which items you have used and which you have not.
\end{itemize}

Your Jansport backpack is a solution to both these problems.

When the backpack is closed, the items inside are hidden from the parser
so they don't interfere with your commands. I recommend keeping it closed
when you are not explicitly managing your inventory.

Also, the backpack compartments allow you to keep your items sorted.
Each pocket can be open or closed independently,
with the same effect of revealing or hiding the items inside from the parser.

Use these however you like, but here's one suggestion:

\begin{itemize}
\item
  Use the book pocket for library books (not other books).
\item
  Use the key pocket for unused keys.
\item
  Use the side pocket for (other) items containing clues
  you'll need to refer to again.
\item
  Use the trash pocket for items you have used but don't want to drop.
  (You can drop most if not all items once you have used them,
  with the exception of two particular keys.)
\item
  Use the main compartment for any other unused items.
\end{itemize}


\hyperref[sec:route-1]{\emph{Return to route}}

\newpage
\subsection{The Cyphered Message}\label{sec:nb-cyphered-message}

In the lion sex book, you find a note with this cyphered message:

\begin{quote}
\begin{verbatim}
BJIYO --
.....

SIX TIF J MOXXMD NIYDXLOBS WHIY XLD CFHOI NLIG.
... ... . ...... ......... .... ... ..... ....

CLDCE XLD LOKKDB GICEDX IW TIFH AJCEGJCE.
..... ... ...... ...... .. .... ........

BIU SI SDX TIFH YJB, SOHM!
... .. ... .... ...  ....

O SIX YOBD JBK UD’HD KIOBS SHDJX.
. ... .... ... .. .. ..... .....

(LD NJTN LO & SIIK MFCE.)
 .. .... ..    .... ....

MIQD, JFBXOD NXDMMJ.
....  ...... ......
\end{verbatim}
\end{quote}

I managed to decypher the message by guessing the letters in this order:

\begin{quote}
\begin{verbatim}
O  J  D  H  U  I  B  Y  W  S  X  K  L  N
.  .  .  .  .  .  .  .  .  .  .  .  .  .

M  F  T  Q  C  G  E  A
.  .  .  .  .  .  .  .
\end{verbatim}
\end{quote}

The solution is on the \hyperref[sec:nb-cyphered-message-sol]{next page}.

Once you have acted on the contents of the message, there is a note about it in
\cref{sec:nb-trolley-system}.

\hyperref[sec:route-1]{\emph{Return to route}}

\newpage\phantomsection\label{sec:nb-cyphered-message-sol}
Decyphered, the message reads as follows:

\begin{quote}
Naomi --

Got you a little something from the curio shop. Check the hidden pocket of your backpack. Now go get your man, girl! I got mine and we’re doing great. (He says hi \& good luck.)

Love,\\
Auntie Stella.
\end{quote}

\hyperref[sec:route-1]{\emph{Return to route}}

\newpage
\subsection{The Trolley System}\label{sec:nb-trolley-system}

The trolley system is a very useful device.
I recommend you wear your pass at all times.

Note that you can type just ‘\textsc{wait for} \emph{colour}’:
the ‘\textsc{line}’ bit is optional.

Locations get added to your schedule as you visit them,
and (as you would expect) you can't use the trolley to get to unvisited locations.
Here is the full list of locations for ease of reference:

\begin{quote}
\begin{description}
\item[Railway Platform] Brown
\item[Church Exterior] Gold
\item[The Old Well] Green
\item[Outside the Library] Blue
\item[River Walk] Aqua
\item[Constabulary Road] Orange
\item[Outside the Plant] Red
\item[Front Walk] Purple
\item[Malign Tunnel] Black
\item[Balcony] Lavender
\item[Attic] Eggplant
\end{description}
\end{quote}

In the route maps in \cref{sec:route}, rooms with trolley stops have thick borders.
The asterisked sets of directions can be replaced or shortened with a trolley journey.

\hyperref[sec:route-1]{\emph{Return to route}}

\newpage
\subsection{The Town Square Emblem}\label{sec:nb-town-square-emblem}

The following might help you imagine what the emblem looks like.
(This is the initial state, not the solution.)

\begin{adjustbox}{center}
\begin{tikzpicture}[baseline=0pt]
\draw[Gray] (0,0) circle[radius=1];
\draw[Gray] (0,0) circle[radius=2];
\draw[Gray] (0,0) circle[radius=3];
\draw[Gray] (0,0) circle[radius=4];

\foreach \x/\xtext in
{90/n, 0/e, 270/s, 180/w}
\node[align=center,font=\sffamily\scshape] at (\x:4.2) {\xtext};

\foreach \x/\xtext in
{90/{Black-\\bird}, 30/Tree, 330/Sparrow, 270/Cube, 210/Woman, 150/Feather}
\node[align=center] at (\x:3.5) {\xtext};

\foreach \x/\xtext in
{90/Daan, 30/Shi, 330/Aak'ee, 270/Hai, 210/Tlèè, 150/Ji}
\node[align=center] at (\x:2.5) {\xtext};

\foreach \x/\xtext in
{90/Pig, 30/Eye, 330/{Wood-\\pecker}, 270/Cross, 210/Fish, 150/Eagle}
\node[align=center] at (\x:1.5) {\xtext};

\node[rotate=-60] at (30:1) {Δ};

\node[align=center] at (0:0) {Bird\\claws};
\end{tikzpicture}
\end{adjustbox}

\bigskip
\hyperref[sec:route-1]{\emph{Return to route}}

\newpage
\subsection{The Riddle from the Worn-Out Picture}\label{sec:nb-picture-riddle}

Here is the riddle on the a worn out, decaying picture:

\begin{quote}
Point the mark towards the cross,\\
Find the eagle a perch,\\
Put its gift over \emph{daan},\\
And you'll soon end your search.\par
Don't forget to say ‘ahe'hee’!
\end{quote}

\hyperref[sec:route-5]{\emph{Return to route}}

\newpage
\subsection{The Music Room}\label{sec:nb-Music-Room}

For your convenience, here is a map of the various locations to which the
podium can take you.

\medskip
\bgroup\tikzset
{ plus/.style  = {maze={act={$+$}}{act={$+$}}}
, minus/.style = {maze={act={$-$}}{act={$-$}}}
, lfo/.style   = {maze={act={L}}{act={L}}}
, tgmnode/.append style={draw=Green}
}
\begin{adjustbox}{valign=t,center}
\sffamily
\begin{gamemap}
\graph
[ no placement
] {
  % 0
  mr/"Music\\Room",
  % 1
  o/"L'oreille" [at=(90:1)],
  gb/"Girl's\\Bedroom" [at=(210:1)],
  lr/"Living\\Room" [at=(330:1)],
  % 2
  pg/"Play-\\ground" [at=(30:1.3)],
  b/"Basement" [at=(150:1.3)],
  cb/"Club\\Backstage" [at=(270:1.3)],
  %3
  fl/"Forest-\\land" [at=(270:2.4)],
  % connections
  mr -- [lfo]
  o -- [minus]
  b -- [lfo]
  gb -- [plus]
  cb -- [minus]
  lr -- [lfo]
  pg -- [plus]
  o,
  mr -- [plus] lr,
  mr -- [minus] gb,
  cb -- [lfo] fl,
  b -- [warp={$+$}{\tgmSW}{$+$}{\tgmW}]
  fl -- [warp={$-$}{\tgmE}{$-$}{\tgmSE}]
  pg
};
\end{gamemap}
\end{adjustbox}

\bigskip
\hyperref[sec:route-6]{\emph{Return to route}}

\newpage
\subsection{The Church Basement}\label{sec:nb-Church-Basement}

\begin{quote}
My dearest Salona:

Apologies, I have still not yet familiarized myself
with this phonographic contraption. I hope this message finds you well, or, as well can be
expected given the circumstances. I must be brief, as I am not certain to tarry long in this fearful
place.

I have instructed my man to send you with this missive your most favorite varietal of
peaches, newly pickled in a jar with a most singular reagent that, I am told, shall restore to you
life and vitality. I only pray this does not reach you too late.

Until next we rejoin, I remain, Edmund---
\end{quote}

\hyperref[sec:route-7]{\emph{Return to route}}

\newpage
\subsection{The Cragne Library Forbidden Annex}\label{sec:nb-Cragne-Library-Forbidden-Annex}

The critical clues from this room that you only get the once:

\vfill
\begin{quote}
Barach’speroth Arguule
\end{quote}

\vfill
\begin{quote}
It must be said three times, in the presence of the carved seal\xelip
that is the only way to banish the spirit and secure what he has stolen\xelip
he is covetous, it is the only way\xelip
but it is so dangerous that I am afraid to do it,
even to retrieve my precious and defiled memory\xelip
the god’s first servant will be summoned\xelip
but I will mark the word, the incantation, before he kills me\xelip
it is the only way\xelip
\end{quote}

\vfill
The remaining critical piece of information, for completeness:
\begin{quote}
ANGARITHEP
\end{quote}

\vfill
\hyperref[sec:route-8]{\emph{Return to route}}

\newpage
\subsection{Pantry}\label{sec:nb-Pantry}

Edmund's poem from the jar of peaches:

\begin{quote}
Under the crescent moons,\\
Beneath the starlit skies,\\
The bells lament with their songs,\\
The bells lament with their cries.

A song of copper and iron,\\
A song of silver and gold,\\
The bells sing of the One's true sign,\\
The bells sing of the Ones of old.
\end{quote}

\hyperref[sec:route-8]{\emph{Return to route}}

\newpage
\subsection{The Walkie-Talkie}\label{sec:nb-walkie-talkie}

The walkie-talkie is another meta device that is intended for you rather than
Naomi.

It allows you to access an author's commentary for the room you are in.
It does not work in every room,
but it's a nice feature for those rooms where it does work.

I leave it as an exercise for the reader to go back through every room and try
it out.

\hyperref[sec:route-10]{\emph{Return to route}}

\newpage
\section{Maps of Backwater, Cragne Manor, and game flow}\label{sec:full-map}

The following maps do not show absolutely all locations in the game.
If an author split their room into several locations, only the main location
is shown, unless the secondary areas have a sufficiently independent character.

The maps have been shrunk down to fit on the page,
so zoom in to see the full detail.

\subsection{Backwater}

\tikzset{tgmnode/.append style={draw=Green}}
\begin{adjustbox}{max width=\linewidth}
\sffamily
\begin{gamemap}[set grid={10em}{6em}]
\graph [no placement] { [x=0]
  "Railway\\Platform" [entry] -- [going=s]
  lobby/"Train Station\\Lobby" [y=-1,exit=e] -- [going=w]
  "Train Station\\Restroom" [x=-1,y=-1],
  lobby -- [going=e]
  "Station\\Security\\Room" [x=1,y=-1],
  lobby -- [going=s]
  "Exterior of\\Train Station" [y=-2]  -- [going=s]
  "Milkweed" [y=-3] -- [going=s]
  ch/"Church\\Exterior" [y=-4,exit=e] -- [going=ne]
  cy/"The\\Churchyard" [x=1,y=-3,exit=ne] -- [going=in]
  "Mausoleum" [x=2,y=-3],
  cy -- [going=ne]
  dr/"The Dim\\Recesses of\\the Forest" [x=2,y=-2] -- [going=se]
  "Shack\\Exterior" [x=3,y=-3] -- [going=se]
  ol/"Outside\\the Library" [x=4,y=-4,exits={e}] -- [going=s]
  ts/"Town\\Square" [x=4,y=-5] -- [going=w]
  ch,
  ol  -- [going=w]
  "Estate\\Agent's\\Office" [x=3,y=-4],
  dr -- [going=n]
  "The Old\\Well" [x=2,y=-1] -- [go={d=\tgmSSE}{u=\tgmNNW}]
  "Circular\\Room" [x=3,y=-2],
  "Drinking\\Fountain" [x=3,y=-6] -- [going=ne]
  ts -- [going=e]
  "Bridge" [x=5,y=-4.7,exit=w] -- [going=e]
  pub/"Outside Pub" [x=6,y=-5,exit=n] -- [going=e]
  cr/"Constabulary\\Road" [x=7,y=-5] -- [going=n]
  "Backwater\\Jail" [x=7,y=-4] -- [going=n]
  "Padded\\Cell" [x=7,y=-3],
  pub -- [going=n]
  "The\\Invisible\\Worm" [x=6,y=-4],
  ts -- [going=se]
  "River\\Walk" [x=5,y=-6] -- [going=n]
  "Under the\\Bridge" [x=5,y=-5.25] -- [go={d=\tgmSE}{u=\tgmNE}]
  st/"Subterranean\\tunnel" [x=5,y=-8] -- [going=nw]
  te/"Tunnel\\Entrance" [x=-2,y=-7] -- [going=ne]
  "Small\\Chamber" [x=-1,y=-6],
  te -- [going=u]
  "Church\\Basement" [x=-2,y=-6] -- [going=u]
  co/"Church\\Office" [x=-2,y=-5],
  cr -- [going=ne]
  hp/"Hillside\\Path" [x=8,y=-4] -- [going=n]
  fw/"Front\\Walk" [x=8,y=-3,exit=n,tunnel] -- [going=nw]
  og/"Outside the\\Greenhouse" [x=7,y=-2] -- [going=ne]
  bg/"The Cragne\\Manor's Back\\Garden" [x=8,y=-1] -- [going=se]
  cfp/"Cragne\\Family\\Plot" [x=9,y=-2] -- [going=sw]
  fw,
  og -- [go={in=\tgmWSW}{o=\tgmENE}]
  "Greenhouse" [x=6,y=-2],
  bg -- [going=in]
  "The\\Shambolic\\Shack" [x=9.5,y=-1],
  cfp -- [going=in]
  "Crypt" [x=10,y=-2],
  hp -- [going=se]
  "Outside\\the Plant" [x=9,y=-5] -- [going=in]
  mp/"The\\meatpacking\\plant" [x=10,y=-6] -- [going=u]
  "Dusty\\Office" [x=10,y=-5],
  mp -- [going=w]
  "Bathroom\\of the\\Meatpacking\\Plant" [x=8.9,y=-6],
  ch -- [go={in=\tgmWSW}{o=\tgmENE}]
  cn/"Church\\Narthex" [x=-1,y=-4] -- [going=u]
  "Steeple" [x=-1,y=-3],
  cn -- [going=w]
  "Chapel" [x=-2,y=-4] -- [going=s]
  co,
  ol -- [going=e]
  "Backwater\\Public\\Library" [x=5,y=-4],
  cr -- [going=s]
  ct/"Courtyard" [x=7,y=-6] -- [going=e]
  "Curiosity\\Shop" [x=8,y=-6],
  ct -- [going=d]
  at/"Amorphous\\Tunnel" [x=7,y=-8] -- [going=ne]
  "Malign\\Tunnel" [x=8,y=-7,tunnel,exit=u],
  at -- [going=w]
  "Narrow\\Straits" [x=6,y=-8] -- [going=w]
  st
};
\end{gamemap}
\end{adjustbox}

\subsection{Cragne Manor}

\begin{adjustbox}{max width=\linewidth}
\sffamily
\begin{gamemap}[set grid={10em}{6em}]
\graph [no placement] {
  fw/"Front\\Walk" [x=0,y=0,entry,exits={nw,ne,s}] -- [going=n]
  "Mudroom" [x=0,y=1] -- [going=n]
  f/"Foyer" [x=0,y=2,exit=n] -- [going=w]
  "Court" [x=-1,y=2],
  f -- [going=e]
  "Gallery" [x=1,y=2] -- [going=n]
  "Rec\\Room" [x=1,y=3] -- [going=e]
  "Music\\Room" [x=2,y=3],
  f -- [going=n]
  lb/"Landing at\\the Bottom\\of Stairs" [x=0,y=3] -- [going=n]
  dr/"Dining\\Room" [x=0,y=5] -- [going=w]
  "Kitchen" [x=-1,y=5] -- [go={d}{u=\tgmNE}]
  b/"Basement" [x=-2,y=4] -- [going=n]
  "Cold Storage\\Room" [x=-2,y=5],
  b -- [going=s]
  "Boiler\\Room" [x=-2,y=3] -- [going=d]
  "Malign\\Tunnel" [x=-2,y=2,tunnel,exit=sw] -- [going=sw]
  at/"Amorphous\\Tunnel" [x=-3,y=1,exit=w] -- [go={u=\tgmNNW}{d=\tgmSSE}]
  "Courtyard" [x=-4,y=2,exit=n] -- [going=e]
  "Curiosity\\Shop" [x=-3,y=2],
  b -- [going=e]
  "Pantry" [x=-1.2,y=4] -- [going=e]
  "Workroom" [x=-0.4,y=4],
  b -- [going=w]
  "Wine\\Cellar" [x=-3,y=4,exit=w] -- [going=w]
  "Laboratory" [x=-4,y=4],
  lb --[going=u]
  "Top of\\Stairs" [x=2,y=5] -- [going=e]
  hn/"Upstairs\\Hall N" [x=3,y=5] -- [going=s]
  hs/"Hallway\\South" [x=3,y=4] -- [going=e]
  "Library" [x=4,y=4],
  hn -- [going=n]
  "Nursery" [x=3,y=6],
  hn -- [going=e]
  "Master\\Bedroom" [x=4,y=5] -- [going=e]
  "Shadowy\\Closet" [x=5,y=5] -- [go={u=\tgmSSW}{d=\tgmNNE}]
  a/"Attic" [x=5,y=1] -- [going=w]
  bc/"Branching\\Corridor" [x=4,y=1] -- [going=w]
  "Science\\Tower" [x=3,y=1] -- [going=sw]
  an/"Abandoned\\Nursery" [x=2,y=0] -- [going=n]
  "Invasive\\Library" [x=2,y=2] -- [going=se]
  bc -- [going=se]
  "Observatory" [x=5,y=0],
  a -- [rounded corners,tgmedge,to path={ -- (5,0.5) node [transit node,pos=0] {S} -- (2.25,0.5) -- (1.75,1.5) -- (1,1.5) -- (\tikztotarget) node [transit node,pos=1] {N} }]
  "Disheveled\\Studio" [x=1,y=1] -- [going=se]
  an,
  hs -- [going=w]
  "Study" [x=2,y=4],
  hs -- [going=s]
  "Balcony" [x=3,y=3],
  dr -- [going=e]
  "Sitting\\Room" [x=1,y=5],
};
\end{gamemap}
\end{adjustbox}

\subsection{Order in which rooms should be completed}

\begin{adjustbox}{max size={\linewidth}{\dimexpr\textheight - 2.5em\relax},center}
\sffamily
\begin{gamemap}[>={Stealth[round,black]},set grid={8.5em}{5.5em}]
\graph [no placement] {
  % Unlocking
  "Train station\\restroom" [x=0,y=26] -> "Exterior of\\Train station" [x=0,y=25] ->[forked=0.55]
  { [y=24] "Milkweed"[x=-3,label={270:\faUniversity}], "The Dim\\Recesses of\\the Forest"[x=-2,label={270:\faUniversity}],
    "Under the\\Bridge (clue)"[x=-1,label={270:\faUniversity}], cy/"Churchyard"[x=0],
    "Shack\\Exterior\\(book)"[x=1,label={0:\faBook}], "Estate\\Agent's\\Office"[x=2,label={0:\faBook}],
    rw/"River\\Walk"[x=-5] },
  cy -> { [y=23] "Mausoleum"[x=1,label={0:\faBook}], "Bridge" [x=0]},
  rw -> { [x=-5] "Railway\\Platform"[y=23] -> "The Old\\Well"[y=22] -> cr/"Circular\\Room"[y=21] },
  "Bridge" ->[forked=0.6]
  { [y=22] "Outside\\Pub"[x=-4,label={270:\faUniversity}], "Constabulary\\Road"[x=1],
    bj/"Backwater\\Jail"[x=-2,label={270:\faBook}], hp/"Hillside\\Path (partial)"[x=3], "Front\\Walk"[x=4],
    "Outside the\\Greenhouse"[x=5], bg/"Back\\Garden"[x=0], op/"Outside\\the Plant"[x=2],
    "Cragne Family\\Plot (book)"[x=-1,label={270:\faBook}]},
  bj <-> "Padded\\Cell"[x=-3,y=22,label={270:\faUniversity}],
  op ->
  { [y=21] "Dusty\\Office"[x=2,label={270:\faUniversity}],
    "Bathroom\\of the Meat-\\packing Plant"[x=1,label={270:\faUniversity}] },
  bg -> "Shambolic\\Shack\\(key 1)"[y=21] -> "Under the\\Bridge (door)"[y=20] ->
  { [y=19] "Tunnel\\Entrance"[x=-1] -> "Small\\Chamber"[x=-1,y=18,label={0:\faBook}],
    "Church\\Office (key)"[x=0,label={0:\faUniversity}] -> mr/"Mudroom"[x=0,y=18] },
  mr ->[forked=0.55]
  { [y=17] "Foyer"[x=-2], c1/"Court\\(key)"[x=-4], "Gallery"[x=-1] -> "Music\\Room"[x=-1,y=16,label={0:\faUniversity}],
    "Landing at\\the Bottom\\of Stairs"[x=1,label={270:\faBook}], "Dining\\Room"[x=-3],
    "Top of\\Stairs"[x=2,label={270:\faBook}], sy/"Study"[x=0], uh1/"Upstairs\\Hall N\\(N door)"[x=5],
    "Hallway\\S (book)"[x=3,label={270:\faBook}],
    "Library"[x=4,label={270:\faBook}] },
  { [x=-4] c1 -> "Cragne Family\\Plot (door)"[y=16] -> "Crypt"[y=15,label={0:\faUniversity}] ->
    cx/"Church\\Exterior"[y=14]} ->
  { ch/"Chapel"[x=-5,y=12.5], "Lobby/\\Narthex"[x=-4,y=13,label={0:\faBook}] -> s1/"Steeple\\(key)"[x=-4,y=11] },
  "TW/Church\\Office (door)"[x=-3,y=14] -- [double] cx,
  sy -> k/Kitchen [x=0,y=16] ->[forked]
  { [y=15] "Basement"[x=1,label={270:\faBook}], "Cold\\Storage"[x=-1,label={270:\faUniversity}], "Workroom"[x=2,label={270:\faBook}], "Boiler\\Room"[x=3,label={270:\faBook}],
    "Malign\\Tunnel"[x=-2], "Courtyard"[x=-3], wc/"Wine\\Cellar"[x=0] ->[hooked] lab/Laboratory [x=-1,y=12]},
  uh1 ->
  { [x=4.5] nc/"Nursery/\\Carol's Room"[y=16] -> "Town\\Square"[y=15] -> "Church\\Basement"[y=14] } ->
  { [x=4] py/"Pantry"[y=13] -> s2/"Steeple\\(stars)"[y=11] },
  k ->[forked=2.5] py,
  hp ->[hooked,dashed,tgmedge] nc,
  "Hillside\\Path"[x=3.5,y=16] <-> nc,
  wc -> "Hallway\\S (door)"[y=14] ->  Balcony [y=13] -> "Upstairs\\Hall N\\(E door)" [y=12,label={0:\faUniversity}] ->
  { [y=11] "Master\\Bedroom"[x=1], sc/"Shadowy\\Closet"[x=0] },
  cr -> ch ->[tgmedge] lab -> sc ->[forked]
  { [y=10] a/"Attic"[x=0], ds/"Disheveled\\Studio"[x=-1,label={0:\faBook}], "Invasive\\Library"[x=1,label={270:\faBook}],
    bc/"Branching\\Corridor"[x=-2], st/"Science\\Tower"[x=-3] },
  a ->
  { [x=0] "Shambolic\\Shack\\(key 2)"[y=9] -> "Rec\\Room"[y=8] -> mpp/"The\\Meatpacking\\Plant"[y=7] ->
    "The\\Invisible\\Worm"[y=6] -> "Greenhouse"[y=5] -> "Abandoned\\Nursery"[y=4] -> "Sitting\\Room"[y=3] ->
    is/"Inside\\the Shack"[y=2] -> obs/"Observatory" [x=0,y=1] },
  op ->[hooked=14,tgmedge,rounded corners,%
        to path={-- (2.5,21) -- (2.5,9) -- (\tikztotarget) \tikztonodes}] mpp,
  ds ->
  { [x=-1] "Train\\Station\\Lobby"[y=5] -> "Station\\Security\\Room"[y=4] -> "Shack\\Exterior\\(door)"[y=3] -> is },
  st -> { [x=-3] "Curiosity\\Shop" [y=9] -> ns/"Narrow\\Straits" [y=8] },
  s1 -> sub/"Subterranean\\Tunnel" [x=-4,y=9] -- [double] "Amorphous\\Tunnel" [x=-5,y=9],
  sub -> ns,
  % Books
  bk1/"9~\faBook"[x=3,y=7] -> "Drinking\\Fountain" [x=2,y=6,label={0:\faBook}] -> pl/"Backwater\\Public\\Library" [x=2,y=3]
      ->[rounded corners, to path={-- (2,2) -- (\tikztotarget) \tikztonodes}] obs,
  bk2/"17~\faBook"[x=3,y=4] -> pl,
  % Variegated Court
  vc/"12~\faUniversity"[x=1,y=4] -> c2/"Court\\(obelisk)" [x=1,y=3]
     ->[rounded corners, to path={-- (1,2) -- (\tikztotarget) \tikztonodes}] obs,
  bc ->[rounded corners, to path={-- (-2,2) -- (\tikztotarget) \tikztonodes}] obs,
  ns ->[rounded corners, to path={-- (-3,2) -- (\tikztotarget) \tikztonodes}] obs,
  s2 ->[rounded corners, to path={-- (4,2) -- (\tikztotarget) \tikztonodes}] obs
};
\end{gamemap}
\end{adjustbox}

\newpage
When using these maps to construct your own route,
here are some points that did not fit neatly on the diagrams:

\begin{itemize}
\item
  Having gained access to the Court and followed the branch down to get the key
  from the Steeple, it is possible to unlock the Subterranean and Amorphous Tunnels
  and thereby gain access to the Manor basement area before visiting the Study
  and the Kitchen. (And you can unlock the Kitchen trap door from below.)

  I didn't do this in my recommended route as I think it makes less sense from
  a narrative perspective.

\item
  If you wait until you have access to the Balcony, then you can use the Trolley
  to cut down on the travel you need to do between the Hillside Path and the
  Nursery\slash Carol's Room, especially if you want to limit the amount of
  supernatural assistance you use.

  In my route I didn't do this because, by completing the pair of rooms early,
  it saves additional trips to mop up the Church Basement and the Pantry.
\end{itemize}

\newpage
\section{Acknowledgements}\label{sec:ack}

In order to complete the game and compile this guide, I relied on both the
questions and answers from the good folk on the Interactive Fiction Community
Forum, specifically these two threads:

\begin{itemize}
\item
  \href{https://intfiction.org/t/cragne-manor-hint-thread/13961}{Cragne Manor hint thread}
\item
  \href{https://intfiction.org/t/cragne-manor-complete-guide-and-walkthrough/14019}{Cragne Manor Complete Guide and Walkthrough}
\end{itemize}

This is not the Complete Guide and Walkthrough, as you may have noticed from the lack of completeness.
\end{document}
